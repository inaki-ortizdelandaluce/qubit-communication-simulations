In the following sections we will discuss the methods applied to implement the prepare-and-measure classical simulation protocols described in section \ref{section:protocols}.
\subsubsection{Classical transmission of one qubit}
So far we have described the methods available to generate random qubit states and random measurements proportional to rank-1 projectors. Once these are available, we should convert them to the corresponding elements in the Bloch sphere, i.e. to vectors $\vec{x}$, ${\vec{y}_k} \in \mathbb{R}^3$, as per protocol description (see section \ref{section:protocol_pm}). 

A qubit state in density matrix form $\rho$ can be easily be transformed into the dual Bloch vector $\vec{x}$ with components $\{x_k\}$ by applying the equation
\begin{equation}
    x_k = tr(\rho \cdot \sigma_k),\ \text{where}\ \vec{\sigma} = (\sigma_x, \sigma_y, \sigma_z)
\end{equation}
Similarly, the vector ${\vec{y}_k}$ associated to the POVM will be the Bloch vector corresponding to the rank-1 projector $ \ket{\vec{y}_k} \bra{\vec{y}_k}$, so the same procedure could be flawlessly applied.

For every random state and POVM, we will then sample the shared randomness $\vec{\lambda}_1$, $\vec{\lambda}_2 \in \mathbb{R}^3$ following a uniform distribution and will apply steps 1 to 4 in the protocol such that for every run we get the probability for each measurement outcome. Eq. (\ref{eq:prob_classic}) can then be computed by just using the probabilities outcomes as weights in a random choice whose outcome gets accumulated for each shared randomness run. The accumulated random choices will lead to the final probabilities which will be then compared against the ones obtained with either the theoretical probabilities as per Born's rule, or the probabilities obtained executing the associated quantum circuit in a quantum simulator (see section \ref{section:quantum_circuit}).

\subsubsection{Quantum circuit counterpart}\label{section:quantum_circuit}
Following Neumark's theorem described in section \ref{section:neumark}, we can now implement any POVM measure of $N$ elements in a quantum circuit by applying the $N\times N$ unitary matrix $U$ resulting from Neumark's theorem to an initial state made of the original qubit state we want to measure $\ket{\Psi}$, together with a set of $n-1$ ancillary qubits, a.k.a. \textit{ancillas}, in a zero state $\ket{0}$ such that $2^n \ge N$ (see Fig. \ref{fig:quantum_circuit}) 

\begin{figure}[!ht]
\centering
\def\myvdots{\ \vdots\ }
\begin{quantikz}
    \lstick[wires=4]{$\mathcal{H}^{\otimes n}$}
      && \lstick{$\ket{\Psi}$}  & \gate[4, nwires=3][2cm]{U} & \meter{} \\
      && \lstick{$\ket{0}$}  & & \meter{} & \rstick[wires=3]{$\mathcal{H}^{\otimes n-1}\ \text{ancilla qubits}$}\\
      && \lstick{\myvdots} & & \myvdots &\\
      && \lstick{$\ket{0}$}  & & \meter{} & 
\end{quantikz}
\caption{Quantum circuit implementing a POVM measure of $N$ outcomes following Neumark's extension theorem.}
\label{fig:quantum_circuit}
\end{figure}

As we can see in the diagram, we will obtain the different probabilities for each of the $N$ possible outcomes of the POVM, by adding an $n$ bit classical register to the circuit, which will then perform classical measures on the circuit's computational basis. Each outcome from a total set of $2^n$ possible outcomes of the classical register will therefore correspond to a POVM element measurement outcome. As an example, the measurement of a qubit state $\ket{\Psi}$ with a 4-element POVM as in Fig. \ref{fig:sic_povm}, will be encoded in a quantum circuit as a $4 \cross 4$ unitary matrix applied to the qubit state $\ket{\Psi}$ plus an ancillary qubit $\ket{0}$ such that all possible outcomes from the classical 2-bit register $\{00, 01, 10, 11\}$, will correspond to the measurement outcomes for each POVM element ($N=4$).

As we will show in section \ref{section:results}, these circuits will be implemented with Qiskit \cite{Qiskit}, IBM's open-source Quantum Computing SDK, and will be run using IBM's quantum processors to obtain the experimental probability distributions to be compared with the results from the classical simulations.

In order to translate the POVM's unitary matrices into universal quantum gates available in the underlying IBM's quantum processors, we could either follow Nielsen and Chuang's textbook \cite{nielsen2000}, and decompose the unitary into a sequence of two-level unitary gates, or rely on the Qiskit's transpiler, which translates any generic circuit into an optimized circuit using a backend's native gate set, allowing users to program for any quantum processor or processor architecture with minimal inputs. For the sake of simplicity we will follow on using Qiskit's transpiler. 
