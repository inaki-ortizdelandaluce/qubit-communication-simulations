\section{Source Code Listings}\label{section:source}
\subsection{qubit.py}\label{section:listing_qubit}
This module provides basic operations for qubit pure states either in two-dimensional the Hilbert space $\mathcal{H}$ or in the Bloch sphere $S_2$. 
\begin{minted}[
frame=lines,
framesep=2mm,
baselinestretch=1.2,
fontsize=\footnotesize,
breaklines
]{python}
import numpy as np
import cmath
import math
from enum import Enum


class Axis(Enum):
    X = 0
    Y = 1
    Z = 2


X = np.array([[0, 1], [1, 0]])
Y = np.array([[0, -1.j], [1j, 0]])
Z = np.array([[1, 0], [0, -1]])
paulis = np.array([X, Y, Z])


class Qubit:
    def __init__(self, ket=np.array([1, 0])):
        """
        Initializes a qubit in the computational basis. If no arguments are provided, it returns the zero state.

        Parameters
        ---------
        ket : ndarray
            The qubit components in the computational basis in a 1-d complex array.
        """
        self.alpha = complex(ket[0])
        self.beta = complex(ket[1])
        self.normalize()

    def __repr__(self):
        return '{} |0> + {} |1>'.format(self.alpha, self.beta)

    def ket(self):
        return np.array([self.alpha, self.beta], dtype=np.complex_)

    def normalize(self):
        arr = self.ket()
        self.alpha, self.beta = arr/np.linalg.norm(arr)

    def rho(self):
        """
        Returns the density matrix corresponding to the qubit in a pure state.

        Returns
        -------
        ndarray
            A 2x2 density matrix corresponding to the qubit in a pure state.
        """
        return np.outer(self.ket(), self.ket().conj())

    def bloch_angles(self):
        """
        Returns the spherical coordinates of the qubit in the Bloch sphere, with polar and azimuthal angles in radians.

        Returns
        -------
        (float, float)
            The Bloch sphere coordinates, first the polar angle and then the azimuthal angle (both in radians).
        """
        r0, phi0 = cmath.polar(self.alpha)
        r1, phi1 = cmath.polar(self.beta)
        theta = 2 * math.acos(r0)
        phi = phi1 - phi0

        return theta, phi

    @staticmethod
    def density2bloch(rho):
        """
        Returns the cartesian coordinates of the specified qubit state in the Bloch sphere.

        Parameters
        ---------
        rho : ndarray
            The qubit state in density matrix form

        Returns
        -------
        (float, float, float)
            The cartesian coordinates of the qubit in the Bloch sphere (xyz).

        """
        # cast complex to real to avoid throwing ComplexWarning, imaginary part should always be zero
        return [np.real(np.trace(np.matmul(rho, sigma))) for sigma in paulis]

    def bloch_vector(self):
        """
        Returns the cartesian coordinates of the qubit in the Bloch sphere.

        Returns
        -------
        (float, float, float)
            The cartesian coordinates of the qubit in the Bloch sphere (xyz).

         """
        return Qubit.density2bloch(self.rho())

    def rotate(self, axis: Axis, angle):
        """
        Rotates the qubit along the specified axis by the specified angle in radians in counterclockwise direction.

        Parameters
        ---------
        axis : Axis

        angle : float
            Angle in radians
        """
        r = math.cos(angle/2) * np.eye(2, 2) - 1.j * math.sin(angle/2) * paulis[axis.value]
        self.alpha, self.beta = r @ self.ket()
        return None
\end{minted}
\newpage
\subsection{measurement.py}\label{section:listing_measurement}
This module implements projection-valued measures and rank-1 positive operator-valued measures for qubit states. 
\begin{minted}[
frame=lines,
framesep=2mm,
baselinestretch=1.2,
fontsize=\footnotesize,
breaklines
]{python}
import numpy as np
import scipy
from qt.qubit import Qubit


class PVM:
    def __init__(self, proj):
        """
        Creates a PVM with the specified rank-1 projectors.

        Parameters
        ---------
        proj : ndarray
            A 3-d array with the constituting rank-1 projectors.
       """
        # check input
        if not np.allclose(np.identity(2), np.sum(proj, axis=0)):
            raise ValueError('PVM projectors do not sum up the identity')

        self.proj = proj
        self.bloch = np.asarray([Qubit.density2bloch(p) for p in proj])

    @classmethod
    def new(cls, qubit: Qubit):
        """
        Creates a PVM with the rank-1 projectors corresponding to the specified qubit state.

        Parameters
        ---------
        qubit : Qubit
            The specified qubit state from which the two rank-1 projectors are generated.
        """
        rho = qubit.rho()
        sigma = np.identity(2) - rho
        proj = np.array([rho, sigma])
        return cls(proj)

    def projector(self, index):
        """
        Returns rank-1 projector for the corresponding index

        Parameters
        ---------
        index : the projector index

        Returns
        -------
        ndarray
            A 2-d array with the corresponding projector.
        """
        return self.proj[index]

    def probability(self, qubit: Qubit):
        """
        Returns the probabilities of the different outcomes for a given qubit state

        Parameters
        ---------
        qubit : the qubit state

        Returns
        -------
        ndarray
            The probabilities for each outcome given the input state, stored in a 1-d array.
        """

        # repeat density matrix along zero axis
        rho = qubit.rho()
        rho = np.repeat(rho[np.newaxis, :, :], self.proj.shape[0], axis=0)

        # compute trace of projectors by density matrix
        return np.real(np.trace(np.matmul(self.proj, rho), axis1=1, axis2=2))


class POVM:

    def __init__(self, weights, proj):
        """
        Creates a POVM with the specified weights and rank-1 projectors.

        Parameters
        ---------
        weights : ndarray
            The positive coefficients of the constituting rank-1 projectors.

        proj : ndarray
            A 3-d array with the constituting rank-1 projectors.
        """
        self.weights = weights
        self.elements = proj * weights[:, np.newaxis, np.newaxis]

        # check input
        if not np.allclose(np.identity(2), np.sum(self.elements, axis=0)):
            raise ValueError('POVM elements do not sum up the identity')

        positive = [np.all(np.linalg.eig(element)[0] >= -np.finfo(np.float32).eps) for element in self.elements]
        if not np.all(positive):
            raise ValueError('Some POVM elements are not definite positive')

        self.bloch = v = np.asarray([Qubit.density2bloch(p) for p in proj])

    @classmethod
    def new(cls, qubits):
        """
        Creates a POVM with the rank-1 projectors corresponding to the specified qubit states.

        Parameters
        ---------
        qubits : ndarray
           The specified array of N-2 qubit states from which the N rank-1 POVM projectors are generated.
        """
        # last element normalizes all POVM elements
        rhos = np.asarray([q.rho() for q in qubits])
        e = np.identity(2) - np.sum(rhos, axis=0)

        # diagonalize last element to obtain remaining rank-1 projectors
        _, w = np.linalg.eig(e)
        q1 = Qubit(w[:, 0])
        q2 = Qubit(w[:, 1])
        qubits = np.append(qubits, np.array([q1, q2]), axis=0)

        # compute POVM weights and elements as a linear program (see Sentís et al. 2013)
        v = np.asarray([q.bloch_vector() for q in qubits])
        n = len(qubits)

        a = np.vstack((np.ones((n,)), v.T))
        b = np.append(np.array([2]), np.zeros(3, ), axis=0)

        # c = np.zeros(n, ) finds a solution instead of minimizing a function
        # lower bounds set to 0.01, this could be fine-tuned
        lp = scipy.optimize.linprog(np.zeros(n, ), A_eq=a, b_eq=b, bounds=(0.01, 1), method='highs')
        _a, _e = lp['x'], np.asarray([q.rho() for q in qubits])

        return cls(_a, _e)

    def element(self, index):
        """
        Returns the POVM element for the corresponding index

        Parameters
        ---------
        index : the POVM element index

        Returns
        -------
        ndarray
            A 2-d array with the corresponding POVM element.
        """
        return self.element[index]

    def probability(self, qubit: Qubit):
        """
        Returns the probabilities of the different outcomes for a given qubit state

        Parameters
        ---------
        qubit : the qubit state

        Returns
        -------
        ndarray
            The probabilities for each outcome given the input state, stored in a 1-d array.
        """

        # repeat density matrix along zero axis
        rho = qubit.rho()
        rho = np.repeat(rho[np.newaxis, :, :], self.elements.shape[0], axis=0)

        # compute trace of projectors by density matrix
        return np.real(np.trace(np.matmul(self.elements, rho), axis1=1, axis2=2))

    def size(self):
        """
        Returns the number of POVM elements

        Returns
        -------
        int
            The number of POVM elements.
        """
        return np.size(self.elements, axis=0)

    def unitary(self):
        """
        Returns the associated unitary matrix in the extended Hilbert space according to Neumark's theorem

        Returns
        -------
        ndarray
            The nxn unitary matrix where n is the number of POVM elements.

        """
        d = 2
        n = self.size()
        u = np.zeros((n, n), dtype=np.complex_)

        # compute the kets of the rank-1 POVM projectors and assign to first d columns
        # v, _, _ = np.linalg.svd(self.elements, full_matrices=True, compute_uv=True, hermitian=False)
        # u[:, 0:d] = v[:, :, 0] / np.linalg.norm(v[:, :, 0], axis=0)
        w, v = np.linalg.eig(self.elements)
        v = v[np.where(w != 0)]
        u[:, 0:d] = v / np.linalg.norm(v, axis=0)

        # remaining n-d columns should correspond to orthogonal projectors in extended space
        p = np.eye(n, dtype=np.complex_)
        for idx in range(d):
            p -= np.outer(u[:, idx], u[:, idx].conj())

        counter = 0
        for b in np.eye(n, dtype=np.complex_):
            w = np.matmul(p, b)
            if not np.isclose(w, 0.0).all():
                w /= np.linalg.norm(w)
                u[:, counter + d] = w
                p -= np.outer(w, w.conj())
                counter += 1
            if counter == (n - d):
                break

        if not np.allclose(np.matmul(u, u.conj().T), np.eye(n)):
            raise ValueError('Neumark\'s square matrix is not unitary')

        return u
\end{minted}
\newpage
\subsection{random.py}\label{section:listings_random}
This module provides means to create random qubit states and random measurements using object classes available on \ref{section:listing_qubit} and \ref{section:listing_measurement} modules. 
\begin{minted}[
frame=lines,
framesep=2mm,
baselinestretch=1.2,
fontsize=\footnotesize,
breaklines
]{python}
import math
import numpy as np
from qt.qubit import Qubit
from qt.measurement import PVM, POVM


def bloch_vector():
    """
    Generates a normalised vector uniformly distributed on the Bloch sphere

    Returns
    -------
    ndarray
            A normalised vector on the Bloch sphere

    """
    theta, phi = qubit().bloch_angles()
    return np.array([math.sin(theta) * math.cos(phi),
                     math.sin(theta) * math.sin(phi),
                     math.cos(theta)])


def qubit():
    """
    Generates a random qubit.

    Returns
    -------
    Qubit
        A random Qubit.
    """
    # evolve the zero state with a random unitary matrix
    # same as returning first column of random unitary matrix
    u = unitary((2, 2))
    return Qubit(u[:, 0])


def pvm():
    """
    Generates a random projection value measure for a qubit

    Returns
    -------
    PVM
            A projection value measure instance.
    """
    q = qubit()
    measurement = PVM.new(q)
    return measurement


def povm(n):
    """
    Generates a random positive operator value measure for a qubit

    Parameters
    ---------
    n : int
        Number of POVM elements. Must be greater than two.

    Returns
    -------
    PVM
            A positive operator value measure instance.
    """
    if n <= 2:
        raise ValueError('Number of POVM elements must be greater thant two')

    qubits = [qubit() for _ in range(n - 2)]
    measurement = POVM.new(qubits)
    return measurement


def unitary(shape):
    """
    Generates a random unitary matrix with the given shape.

    Parameters
    ---------
    shape : int or tuple of ints
        Shape of the unitary matrix.

    Returns
    -------
    ndarray
        Unitary matrix with the given shape.
    """
    # build random complex matrix
    m = np.random.normal(0, 1, shape) + 1.j * np.random.normal(0, 1, shape)

    # apply Gram-Schmidt QR decomposition to orthogonalize the matrix
    q, *_ = np.linalg.qr(m, mode='complete')

    return q
\end{minted}
\newpage
\subsection{observable.py}\label{section:listings_observable}
This module provides means to obtain the eigenvalues and eigenvector of an observable. 
\begin{minted}[
frame=lines,
framesep=2mm,
baselinestretch=1.2,
fontsize=\footnotesize,
breaklines
]{python}
import numpy as np
import scipy as sp


class Observable:
    def __init__(self, matrix):

        if not sp.linalg.ishermitian(matrix):
            raise ValueError('Input matrix is not hermitian')

        self.matrix = matrix
        self.eigenvalues, self.eigenvectors = np.linalg.eig(matrix)

    def eigen(self):
        return self.eigenvalues, self.eigenvectors

    def eigenvector(self, eigenvalue):
        m = self.eigenvectors[:, np.where(np.isclose(self.eigenvalues, eigenvalue))]
        m = m.reshape(m.size, )
        if m.size == 0:
            return None
        else:
            return m
\end{minted}
\newpage
\subsection{qudit.py}\label{section:listings_qudit}
This module provides means to create a qudit state, and particularly to obtain the qudit corresponding to a bipartite state of two qubits required to compute the probabilities for a Bell scenario as per \ref{section:listings_bell} module. 
\begin{minted}[
frame=lines,
framesep=2mm,
baselinestretch=1.2,
fontsize=\footnotesize,
breaklines
]{python}
import numpy as np
from qt.qubit import Qubit


class Qudit:
    def __init__(self, ket):
        """
        Initializes a qudit in the computational basis.

        Parameters
        ---------
        ket : ndarray
            The qudit components in the computational basis in a 1-d complex array.
        """
        self.ket = ket
        self.normalize()

    def normalize(self):
        self.ket = self.ket/np.linalg.norm(self.ket)

    def rho(self):
        """
        Returns the density matrix corresponding to the qudit in a pure state.

        Returns
        -------
        ndarray
            A nxn density matrix corresponding to the qubit in a pure state.
        """
        return np.outer(self.ket, self.ket.conj())

    @classmethod
    def bipartite(cls, q1: Qubit, q2: Qubit):
        ket = np.tensordot(q1.ket(), q2.ket(), axes=0).reshape(4, )
        return cls(ket)
\end{minted}
\newpage
\subsection{bell.py}\label{section:listings_bell}
This module provides means to compute the probabilities, expected values and CHSH inequality for any Bell state and the set of Alice and Bob observables. 
\begin{minted}[
frame=lines,
framesep=2mm,
baselinestretch=1.2,
fontsize=\footnotesize,
breaklines
]{python}
from enum import Enum
import numpy as np
import math
from qt.observable import Observable
from qt.qudit import Qudit
from qt.qudit import Qubit


class BellState(Enum):
    PHI_PLUS = 0
    PHI_MINUS = 1
    PSI_PLUS = 2
    PSI_MINUS = 3


class BellScenario:

    _states = {
        0: Qudit(1 / math.sqrt(2) * np.array([1, 0, 0, 1])),
        1: Qudit(1 / math.sqrt(2) * np.array([1, 0, 0, -1])),
        2: Qudit(1 / math.sqrt(2) * np.array([0, 1, 1, 0])),
        3: Qudit(1 / math.sqrt(2) * np.array([0, 1, -1, 0]))
    }

    def __init__(self, state: BellState, alice: tuple[Observable, Observable], bob: tuple[Observable, Observable]):

        if type(alice) is not tuple:
            raise ValueError('Alice\'s observables is not a valid tuple')
        elif len(alice) != 2:
            raise ValueError('Alice\'s number of observables is not valid:{}'.format(str(len(alice))))

        if type(bob) is not tuple:
            raise ValueError('Bob\'s observables is not a valid tuple')
        elif len(bob) != 2:
            raise ValueError('Bob\'s number of observables is not tuple:{}'.format(str(len(bob))))

        self.state = self._states[state.value]
        self.alice = alice
        self.bob = bob

    def probability(self):
        p = np.zeros((4, 4), dtype=float)
        for u in range(4):
            for v in range(4):
                i, j, m, n = (-1) ** (u >> 1), (-1) ** (u % 2), v >> 1, v % 2

                a = Qubit(self.alice[m].eigenvector(i))
                b = Qubit(self.bob[n].eigenvector(j))
                ab = Qudit.bipartite(a, b)
                p[u, v] = np.real(np.trace(np.matmul(ab.rho(), self.state.rho())))
                # print('p{}{}=({},{})x(A{},B{})={}'.format(u, v, i, j, m, n, p[u, v]))
        return p

    def expected_values(self):
        p = self.probability()
        sign = np.array([1, -1, -1, 1])
        return np.sum(p * sign[:, np.newaxis], axis=0)

    def chsh(self):
        e = self.expected_values()
        return abs(np.sum(e * np.array([1, 1, 1, -1])))
\end{minted}
\newpage
\subsection{classical.py}\label{section:listings_classical}
This module implements the classical prepare-and-measure and Bell simulation protocols. 
\begin{minted}[
frame=lines,
framesep=2mm,
baselinestretch=1.2,
fontsize=\footnotesize,
breaklines
]{python}
import math
import numpy as np
import random
import qt.random
from qt.qubit import Qubit
from qt.measurement import PVM, POVM
from qt.bell import BellState, BellScenario
from qt.observable import Observable


def heaviside(a):
    if isinstance(a, np.ndarray):
        return (a >= 0).astype(int)
    else:
        return int(a >= 0)


def theta(a):
    return a * heaviside(a)


def prepare(lambdas, qubit):
    """
    Alice prepares and sends two bits to Bob

    Parameters
    ---------
    lambdas : ndarray
        Shared randomness as two normalized vectors in a numpy 2-d array

    qubit: Qubit
        A uniformly sampled pure state qubit

    Returns
    -------
    dict
        A dictionary with the shared randomness ('lambdas'), the random qubit ('qubit')
        and the bits to be communicated to Bob ('bits')
    """

    x = qubit.bloch_vector()
    bits = heaviside(np.matmul(x, lambdas.T))
    return {
        "lambdas": lambdas,
        "qubit": qubit,
        "bits": bits
    }


def measure_pvm(lambdas, bits, measurement: PVM):
    """
    Bob receives two bits from Alice and performs a random PVM

    Parameters
    ---------
    lambdas : ndarray
        Shared randomness as two normalized vectors in a numpy 2-d array

    bits: ndarray
        Bits communicated by Alice in a numpy 1-d array

    measurement: PVM
        A uniformly sampled PVM

    Returns
    -------
    dict
        A dictionary with the random measurement ('measurement') and
        the probabilities for each measurement outcome ('probabilities')
    """
    # flip shared randomness
    flip = np.where(bits == 0, -1, 1).reshape(2, 1)
    lambdas = np.multiply(lambdas, flip)

    # pick bloch vectors from PVM elements
    y = measurement.bloch
    pb = 0.5 * np.array([1, 1])

    # compute lambda according to the probabilities {pb}
    index = random.choices(range(0, 2), cum_weights=np.cumsum(pb), k=1)[0]
    a = np.abs(np.matmul(lambdas, y.T))
    _lambda = lambdas[np.argmax(a, axis=0)[index]]

    # compute probabilities
    thetas = theta(np.matmul(y, _lambda.reshape(-1, 1)))
    weighted_thetas = np.multiply(thetas, pb.reshape(-1, 1))
    p = weighted_thetas[:, 0] / np.sum(weighted_thetas, axis=0)

    return {
        "measurement": measurement,
        "probabilities": p
    }


def prepare_and_measure_pvm(shots):
    """
    Runs a prepare-and-measure classical simulation with a random PVM measurement

    Parameters
    ---------
    shots : int
        Number of shots the simulation is run with

    Returns
    -------
    dict
        A dictionary with the random state ('qubit'), random PVM measurement ('measurement') and
        the probabilities for each measurement outcome ('probabilities') in a nested structure including the
        theoretical probability ('born'), the execution runs ('runs') and the probability statistics ('stats')
    """

    # Alice prepares a random qubit
    qubit = qt.random.qubit()

    # Bob prepares a random measurement
    measurement = qt.random.pvm()

    experiment = {
        "qubit": qubit,
        "measurement": measurement,
        "probabilities": {
            "runs": np.zeros((shots, 2)),
            "stats": np.zeros((2,)),
            "born": np.ones((2,))
        }
    }

    for i in range(shots):

        # Alice and Bob's shared randomness
        shared_randomness = np.array([qt.random.bloch_vector(), qt.random.bloch_vector()])

        # Alice prepares
        alice = prepare(shared_randomness, qubit)

        # Bob measures
        bob = measure_pvm(shared_randomness, alice['bits'], measurement)

        # save simulation runs
        p = np.abs(bob['probabilities'])
        experiment['probabilities']['runs'][i, :] = p

        # accumulate counts according to Bob's probabilities
        index = random.choices(range(0, 2), cum_weights=np.cumsum(p), k=1)[0]
        experiment['probabilities']['stats'][index] = experiment['probabilities']['stats'][index] + 1

    experiment['probabilities']['stats'] = experiment['probabilities']['stats'] / shots
    experiment['probabilities']['born'] = bob['measurement'].probability(qubit)

    return experiment


def measure_povm(lambdas, bits, measurement: POVM):
    """
    Bob receives two bits from Alice and performs a random POVM

    Parameters
    ---------
    lambdas : ndarray
        Shared randomness as two normalized vectors in a numpy 2-d array

    bits: ndarray
        Bits communicated by Alice in a numpy 1-d array

    measurement: POVM
        A uniformly sampled POVM

    Returns
    -------
    dict
        A dictionary with the random measurement ('measurement') and
        the probabilities for each measurement outcome ('probabilities')
    """
    # flip shared randomness
    flip = np.where(bits == 0, -1, 1).reshape(2, 1)
    lambdas = np.multiply(lambdas, flip)

    # pick bloch vectors and weights from POVM elements
    y = measurement.bloch
    pb = measurement.weights / 2.

    # compute lambda according to the probabilities {pb}
    index = random.choices(range(0, measurement.size()), cum_weights=np.cumsum(pb), k=1)[0]
    a = np.abs(np.matmul(lambdas, y.T))
    _lambda = lambdas[np.argmax(a, axis=0)[index]]

    # compute probabilities
    thetas = theta(np.matmul(y, _lambda.reshape(-1, 1)))
    weighted_thetas = np.multiply(thetas, pb.reshape(-1, 1))
    p = weighted_thetas[:, 0] / np.sum(weighted_thetas, axis=0)

    return {
        "measurement": measurement,
        "probabilities": p
    }


def prepare_and_measure_povm(shots, n=4, qubit=None, measurement=None):
    """
    Runs a prepare-and-measure classical simulation with a random POVM measurement

    Parameters
    ---------
    shots : int
        Number of shots the simulation is run with

    n: int, optional
        Number of POVM random elements. Used if no measurement argument is specified, default value is 4.

    qubit : Qubit, optional
        Alice's qubit state. If not specified, a random qubit state will be used instead

    measurement: POVM, optional
        Bob's POVM measurement. If not specified a random POVM will be used instead


    Returns
    -------
    dict
        A dictionary with the random state ('qubit'), random POVM measurement ('measurement') and
        the probabilities for each measurement outcome ('probabilities') in a nested structure including the
        theoretical probability ('born'), the execution runs ('runs') and the probability statistics ('stats')
    """

    if qubit is None:
        # Alice prepares a random qubit
        qubit = qt.random.qubit()

    if measurement is None:
        # Bob prepares a random measurement
        measurement = qt.random.povm(n)

    n = measurement.size()

    experiment = {
        "qubit": qubit,
        "measurement": measurement,
        "probabilities": {
            "runs": np.zeros((shots, n)),
            "stats": np.zeros((n,)),
            "born": np.ones((n,))
        }
    }

    for i in range(shots):

        # Alice and Bob's shared randomness
        shared_randomness = np.array([qt.random.bloch_vector(), qt.random.bloch_vector()])

        # Alice prepares
        alice = prepare(shared_randomness, qubit)

        # Bob measures
        bob = measure_povm(shared_randomness, alice['bits'], measurement)

        # save simulation runs
        p = np.abs(bob['probabilities'])
        experiment['probabilities']['runs'][i, :] = p

        # accumulate counts according to Bob's probabilities
        index = random.choices(range(0, n), cum_weights=np.cumsum(p), k=1)[0]
        experiment['probabilities']['stats'][index] = experiment['probabilities']['stats'][index] + 1

    experiment['probabilities']['stats'] = experiment['probabilities']['stats'] / shots
    experiment['probabilities']['born'] = bob['measurement'].probability(qubit)

    return experiment


def bell_singlet_full(shots, alice, bob):
    """
    Runs a classical simulation on a Bell singlet state for a set of local observables

    Parameters
    ---------
    shots : int
        Number of shots the simulation is run with

    alice: tuple[Observable, Observable]
        Alice's local projective measurements described as a tuple of observables

    bob: tuple[Observable, Observable]
        Bob's local projective measurements described as a tuple of observables

    Returns
    -------
    dict
        A dictionary with the Bell state ('state'), Alice's and Bob's local projective measurements ('alice', 'bob')
        and the joint probabilities for each measurement outcome ('probabilities') in a nested structure including the
        theoretical probabilities ('born'), the execution runs ('runs') and the probability statistics ('stats')
    """

    if type(alice) is not tuple:
        raise ValueError('Alice\'s observables is not a valid tuple')
    elif len(alice) != 2:
        raise ValueError('Alice\'s number of observables is not valid:{}'.format(str(len(alice))))

    if type(bob) is not tuple:
        raise ValueError('Bob\'s observables is not a valid tuple')
    elif len(bob) != 2:
        raise ValueError('Bob\'s number of observables is not tuple:{}'.format(str(len(bob))))

    state = BellState.PSI_MINUS

    experiment = {
        "state": state,
        "alice": alice,
        "bob": bob,
        "probabilities": {
            "runs": np.zeros((shots, 4, 4)),
            "stats": np.zeros((4, 4)),
            "born": np.zeros((4, 4))
        }
    }

    # Compute theoretical probabilities
    bell = BellScenario(state, alice, bob)
    experiment['probabilities']['born'] = bell.probability().T

    for i in range(2):

        # Alice's state corresponding to the positive local projector
        a = Qubit(alice[i].eigenvector(1))

        for j in range(2):

            # Bob's states corresponding to the positive and negative local projectors
            b = (Qubit(bob[j].eigenvector(1)), Qubit(bob[j].eigenvector(-1)))

            ab = bell_singlet(shots, a, b)

            ij = int('{}{}'.format(i, j), 2)
            experiment['probabilities']['runs'][:, :, ij] = ab['probabilities']['runs']
            experiment['probabilities']['stats'][ij] = ab['probabilities']['stats']

    return experiment


def bell_singlet(shots, a, b):
    """
    Runs a classical simulation on a Bell singlet state for a set of states corresponding to local projection valued
    measurements

    Parameters
    ---------
    shots : int
        Number of shots the simulation is run with

    a: Qubit
        Alice's state corresponding to the positive local projection valued measurement operator

    b: tuple(Qubit, Qubit)
        Bob's states corresponding to the positive and negative local projection valued measurement operators

    Returns
    -------
    dict
        A dictionary with the joint probabilities for each measurement outcome ('probabilities') in a nested structure
        including the execution runs ('runs') and the probability statistics ('stats')
    """

    experiment = {
        "probabilities": {
            "runs": np.zeros((shots, 4)),
            "stats": np.zeros((4,))
        }
    }

    # Alice's positive local projector as bloch vector
    x = np.asarray([a.bloch_vector()])

    # Bob's local projectors as bloch vectors
    y = np.asarray([b[0].bloch_vector(), b[1].bloch_vector()])

    for i in range(shots):

        # Alice and Bob's shared randomness
        lambda1, lambda2 = qt.random.bloch_vector(), qt.random.bloch_vector()

        # Alice performs local projective measurements
        a = - np.sign(x @ lambda1)

        # Alice sends bit to Bob
        c = -a * np.sign(x @ lambda2)

        # Bob flips the lambda if c = -1
        lambda2 = c * lambda2

        # compute lambda according to the probabilities {pb}
        lambdas = np.array([lambda1, lambda2])

        pb = 0.5 * np.array([1, 1])
        index = random.choices(range(0, 2), cum_weights=np.cumsum(pb), k=1)[0]
        ly = np.abs(np.matmul(lambdas, y.T))
        _lambda = lambdas[np.argmax(ly, axis=0)[index]]

        # compute probabilities
        thetas = theta(np.matmul(y, _lambda.reshape(-1, 1)))
        weighted_thetas = np.multiply(thetas, pb.reshape(-1, 1))
        p = weighted_thetas[:, 0] / np.sum(weighted_thetas, axis=0)

        bits = '{}{}'.format(int(0.5 * (1 - a[0])), np.where(p == 1)[0][0])
        index = int(bits, 2)
        experiment['probabilities']['stats'][index] += 1

    experiment['probabilities']['stats'] = experiment['probabilities']['stats'] / shots

    return experiment
\end{minted}
\newpage
\subsection{quantum.py}\label{section:listings_quantum}
This module implements the prepare-and-measure for a given input state and positive operator-valued measure in a quantum simulator using Qiskit. 
\begin{minted}[
frame=lines,
framesep=2mm,
baselinestretch=1.2,
fontsize=\footnotesize,
breaklines
]{python}
import numpy as np
from qiskit import QuantumCircuit, transpile, Aer, IBMQ
from qt.qubit import Qubit
from qt.measurement import POVM


def prepare_and_measure_povm(shots, qubit: Qubit, povm: POVM):
    # TODO extend usage to any POVM of N elements (currently N =4)
    qc = QuantumCircuit(2, 2)
    u = povm.unitary()

    qc.initialize(qubit.ket(), 0)
    qc.unitary(u, [0, 1])
    qc.measure([0, 1], [0, 1])

    backend = Aer.get_backend('aer_simulator')
    qc_transpiled = transpile(qc, backend)

    job = backend.run(qc_transpiled, shots=shots, memory=True)
    result = job.result()
    counts = result.get_counts(qc_transpiled)

    p = np.array([counts['00'], counts['01'], counts['10'], counts['11']])
    p = p / np.sum(p)

    results = {
        "counts": counts,
        "memory": result.get_memory(),
        "probabilities": p
    }
    return results
\end{minted}
\newpage
\subsection{main.py}\label{section:listing_main}
This is the main program, which runs the classical and quantum simulations as well as other experiments discussed throughout this document. 
\begin{minted}[
frame=lines,
framesep=2mm,
baselinestretch=1.2,
fontsize=\footnotesize,
breaklines
]{python}
import math
import matplotlib as mpl
import matplotlib.pyplot as plt
from healpy.pixelfunc import ang2pix
from scipy.special import rel_entr

import qt.classical
import qt.quantum
import qt.random

from qt.qubit import X, Y, Z, Qubit
from qt.bell import BellScenario, BellState
from qt.measurement import POVM
from qt.observable import Observable
from qt.visualization import *

from qiskit import transpile
from qiskit import execute, Aer, IBMQ
from qiskit.visualization import plot_histogram
from qiskit import QuantumCircuit
from qiskit.tools.monitor import job_monitor


def random_states():
    size = 100000
    n = 4
    pixels = 12 * n ** 2
    indexes = np.zeros(size)
    for i in range(size):
        theta, phi = qt.random.qubit().bloch_angles()
        pix = ang2pix(n, theta, phi)
        indexes[i] = pix

    mpl.rcParams['mathtext.fontset'] = 'stix'
    mpl.rcParams['font.family'] = 'STIXGeneral'

    fig, ax = plt.subplots()

    count, bins, ignored = ax.hist(indexes, bins=range(pixels + 1), density=True, fill=True, facecolor='whitesmoke',
                                   edgecolor='k', hatch='', linewidth=1, histtype='step')
    ax.plot(bins, np.ones_like(bins) / pixels, linewidth=2, color='b', linestyle='-', zorder=2)
    ax.set_xlabel('Pixel indices', labelpad=6)
    ax.set_xticks(np.append(np.arange(0, pixels, 25), pixels))
    ax.set_ylabel('Frequency', labelpad=6)
    # ax.set_yticklabels(ax.get_yticklabels() * pixels)
    ax.set_xlim(0, pixels)
    plt.show()
    return None


def random_povm():
    np.random.seed(0)
    q1 = qt.random.qubit()
    q2 = qt.random.qubit()

    povm = POVM.new(np.array([q1, q2]))
    elements = povm.elements

    for i in range(elements.shape[0]):
        print('\nE{} eigenvalues -> {}'.format(i, np.linalg.eig(elements[i])[0]))
        print('\nE{}=\n{}'.format(i, elements[i]))
        print('E{} >=0 > -> {}'.format(i, (np.all(np.linalg.eig(elements[i])[0] >= -np.finfo(np.float32).eps))))

    print('Sum E_i = I -> {}'.format(np.allclose(np.identity(2), np.sum(elements, axis=0))))

    return None


def pm_pvm(shots):
    # run experiment
    np.random.seed(0)
    experiment = qt.classical.prepare_and_measure_pvm(shots)

    # plot probability convergence
    qubit = experiment['qubit']
    runs = experiment['probabilities']['runs']
    stats = experiment['probabilities']['stats']
    born = experiment['probabilities']['born']
    print('Qubit: {}'.format(str(qubit)))
    print('Stats:\np1={},p2={},pt={}'.format(stats[0], stats[1], np.sum(stats)))
    print('Born:\np1={},p2={},pt={}'.format(born[0], born[1], np.sum(born)))

    p = np.cumsum(runs[:, 0]) / (np.arange(len(runs[:, 0])) + 1)

    plt.plot(p)
    plt.axhline(y=born[0], color='r', linestyle='-')
    plt.show()
    return None


def pm_random(shots):
    # run experiment
    # np.random.seed(1200)
    np.random.seed(0)
    experiment = qt.classical.prepare_and_measure_povm(shots, 4)

    # plot probability convergence
    qubit = experiment['qubit']
    runs = experiment['probabilities']['runs']
    stats = experiment['probabilities']['stats']
    born = experiment['probabilities']['born']
    print('Qubit: {}'.format(str(qubit)))
    print('Stats:\np1={}, p2={}, p3={}, p4={}, pt={}'.format(stats[0], stats[1], stats[2], stats[3], np.sum(stats)))
    print('Born:\np1={}, p2={}, p3={}, p4={}, pt={}'.format(born[0], born[1], born[2], born[3], np.sum(born)))

    p1 = np.cumsum(runs[:, 0]) / (np.arange(len(runs[:, 0])) + 1)
    p2 = np.cumsum(runs[:, 1]) / (np.arange(len(runs[:, 1])) + 1)
    p3 = np.cumsum(runs[:, 2]) / (np.arange(len(runs[:, 2])) + 1)
    p4 = np.cumsum(runs[:, 3]) / (np.arange(len(runs[:, 3])) + 1)

    plt.plot(p1, color='r')
    plt.plot(p2, color='g')
    plt.plot(p3, color='b')
    plt.plot(p4, color='y')
    plt.axhline(y=born[0], color='r', linestyle='-')
    plt.axhline(y=born[1], color='g', linestyle='-')
    plt.axhline(y=born[2], color='b', linestyle='-')
    plt.axhline(y=born[3], color='y', linestyle='-')
    plt.title('Prepare and Measure with random state and POVM')
    plt.show()
    return None


def pm_trine(shots):
    # run experiment
    psi = qt.qubit.Qubit(np.array([(3 + 1.j * math.sqrt(3)) / 4., -0.5]))

    one = Qubit(np.array([1, 0])).rho()
    two = Qubit(0.5 * np.array([1, math.sqrt(3)])).rho()
    three = Qubit(0.5 * np.array([1, -math.sqrt(3)])).rho()
    povm = POVM(weights=2./3 * np.array([1, 1, 1]), proj=np.array([one, two, three], dtype=complex))

    experiment = qt.classical.prepare_and_measure_povm(shots, qubit=psi, measurement=povm)

    # plot probability convergence
    runs = experiment['probabilities']['runs']
    stats = experiment['probabilities']['stats']
    born = experiment['probabilities']['born']
    print('Stats:\np1={}, p2={}, p3={}, pt={}'.format(stats[0], stats[1], stats[2], np.sum(stats)))
    print('Born:\np1={}, p2={}, p3={}, pt={}'.format(born[0], born[1], born[2], np.sum(born)))

    p1 = np.cumsum(runs[:, 0]) / (np.arange(len(runs[:, 0])) + 1)
    p2 = np.cumsum(runs[:, 1]) / (np.arange(len(runs[:, 1])) + 1)
    p3 = np.cumsum(runs[:, 2]) / (np.arange(len(runs[:, 2])) + 1)

    plt.plot(p1, color='r')
    plt.plot(p2, color='g')
    plt.plot(p3, color='b')
    plt.axhline(y=born[0], color='r', linestyle='-')
    plt.axhline(y=born[1], color='g', linestyle='-')
    plt.axhline(y=born[2], color='b', linestyle='-')
    plt.title('Prepare and Measure with Trine POVM')
    plt.show()

    return None


def pm_cross(shots):
    # run experiment
    psi = qt.qubit.Qubit(np.array([(3 + 1.j * math.sqrt(3)) / 4., -0.5]))

    zero = np.array([[1, 0], [0, 0]])
    one = np.array([[0, 0], [0, 1]])
    plus = 0.5 * np.array([[1, 1], [1, 1]])
    minus = 0.5 * np.array([[1, -1], [-1, 1]])

    povm = POVM(weights=0.5 * np.array([1, 1, 1, 1]), proj=np.array([zero, one, plus, minus], dtype=complex))

    experiment = qt.classical.prepare_and_measure_povm(shots, qubit=psi, measurement=povm)

    # plot probability convergence
    runs = experiment['probabilities']['runs']
    stats = experiment['probabilities']['stats']
    born = experiment['probabilities']['born']
    print('Stats:\np1={}, p2={}, p3={}, pt={}'.format(stats[0], stats[1], stats[2], np.sum(stats)))
    print('Born:\np1={}, p2={}, p3={}, pt={}'.format(born[0], born[1], born[2], np.sum(born)))

    p1 = np.cumsum(runs[:, 0]) / (np.arange(len(runs[:, 0])) + 1)
    p2 = np.cumsum(runs[:, 1]) / (np.arange(len(runs[:, 1])) + 1)
    p3 = np.cumsum(runs[:, 2]) / (np.arange(len(runs[:, 2])) + 1)

    plt.plot(p1, color='r')
    plt.plot(p2, color='g')
    plt.plot(p3, color='b')
    plt.axhline(y=born[0], color='r', linestyle='-')
    plt.axhline(y=born[1], color='g', linestyle='-')
    plt.axhline(y=born[2], color='b', linestyle='-')
    plt.title('Prepare and Measure with Cross POVM')
    plt.show()

    return None


def pm_sic(shots):
    # run experiment
    psi = qt.qubit.Qubit(np.array([(3 + 1.j * math.sqrt(3)) / 4., -0.5]))

    one = Qubit(np.array([1, 0])).rho()
    two = Qubit(np.array([1/math.sqrt(3), math.sqrt(2/3)])).rho()
    three = Qubit(np.array([1/math.sqrt(3),
                            math.sqrt(2/3) * (math.cos(2 * math.pi/3) + 1.j * math.sin(2 * math.pi/3))])).rho()
    four = Qubit(np.array([1/math.sqrt(3),
                           math.sqrt(2/3) * (math.cos(4 * math.pi/3) + 1.j * math.sin(4 * math.pi/3))])).rho()

    povm = POVM(weights=0.5 * np.array([1, 1, 1, 1]), proj=np.array([one, two, three, four], dtype=complex))

    experiment = qt.classical.prepare_and_measure_povm(shots, qubit=psi, measurement=povm)

    # plot probability convergence
    runs = experiment['probabilities']['runs']
    stats = experiment['probabilities']['stats']
    born = experiment['probabilities']['born']
    print('Stats:\np1={}, p2={}, p3={}, p4={}, pt={}'.format(stats[0], stats[1], stats[2], stats[3], np.sum(stats)))
    print('Born:\np1={}, p2={}, p3={}, p4={}, pt={}'.format(born[0], born[1], born[2], born[3], np.sum(born)))

    p1 = np.cumsum(runs[:, 0]) / (np.arange(len(runs[:, 0])) + 1)
    p2 = np.cumsum(runs[:, 1]) / (np.arange(len(runs[:, 1])) + 1)
    p3 = np.cumsum(runs[:, 2]) / (np.arange(len(runs[:, 2])) + 1)
    p4 = np.cumsum(runs[:, 3]) / (np.arange(len(runs[:, 3])) + 1)

    plt.plot(p1, color='r')
    plt.plot(p2, color='g')
    plt.plot(p3, color='b')
    plt.plot(p4, color='y')
    plt.axhline(y=born[0], color='r', linestyle='-')
    plt.axhline(y=born[1], color='g', linestyle='-')
    plt.axhline(y=born[2], color='b', linestyle='-')
    plt.axhline(y=born[3], color='y', linestyle='-')
    plt.title('Prepare and Measure with SIC-POVM of 4 elements')
    plt.show()
    return None


def neumark():
    psi = qt.qubit.Qubit(np.array([(3 + 1.j * math.sqrt(3)) / 4., -0.5]))
    print(psi.bloch_vector())

    zero = np.array([[1, 0], [0, 0]])
    one = np.array([[0, 0], [0, 1]])
    plus = 0.5 * np.array([[1, 1], [1, 1]])
    minus = 0.5 * np.array([[1, -1], [-1, 1]])

    povm = POVM(weights=0.5 * np.array([1, 1, 1, 1]), proj=np.array([zero, one, plus, minus], dtype=complex))
    unitary = povm.unitary()
    print(unitary)
    return None


def pm_circuit():
    qubit = qt.qubit.Qubit(np.array([(3 + 1.j * math.sqrt(3)) / 4., -0.5]))

    zero = np.array([[1, 0], [0, 0]])
    one = np.array([[0, 0], [0, 1]])
    plus = 0.5 * np.array([[1, 1], [1, 1]])
    minus = 0.5 * np.array([[1, -1], [-1, 1]])
    povm = POVM(weights=0.5 * np.array([1, 1, 1, 1]), proj=np.array([zero, one, plus, minus], dtype=complex))
    print(povm.unitary())

    shots = 10 ** 7
    results = qt.quantum.prepare_and_measure_povm(shots, qubit, povm)
    print('Probabilities={}'.format(results["probabilities"]))
    return None


def quantum_simulator():
    qc = QuantumCircuit(2, 2)

    psi = ((3 + 1.j * math.sqrt(3)) / 4., -0.5)

    U = [[0.70710678 + 0.j, 0. + 0.j, 0.70710678 + 0.j, 0. + 0.j],
         [0. + 0.j, 0.70710678 + 0.j, 0. + 0.j, 0.70710678 + 0.j],
         [0.5 - 0.j, 0.5 + 0.j, -0.5 + 0.j, -0.5 + 0.j],
         [-0.5 + 0.j, 0.5 + 0.j, 0.5 + 0.j, -0.5 + 0.j]]

    qc.initialize(psi, 0)
    qc.unitary(U, [0, 1])
    qc.measure([0, 1], [0, 1])
    qc.draw()

    backend = Aer.get_backend('aer_simulator')
    qc_transpiled = transpile(qc, backend)
    qc_transpiled.draw()

    job = backend.run(qc_transpiled, shots=4000)
    result = job.result()
    counts = result.get_counts(qc_transpiled)

    print(counts)
    plot_histogram(counts)

    sum(counts.values())
    print(counts['00'] / sum(counts.values()))
    print(counts['01'] / sum(counts.values()))
    print(counts['10'] / sum(counts.values()))
    print(counts['11'] / sum(counts.values()))


def quantum_computer():
    qc = QuantumCircuit(2, 2)

    psi = ((3 + 1.j * math.sqrt(3)) / 4., -0.5)

    U = [[0.70710678 + 0.j, 0. + 0.j, 0.70710678 + 0.j, 0. + 0.j],
         [0. + 0.j, 0.70710678 + 0.j, 0. + 0.j, 0.70710678 + 0.j],
         [- 0.5 + 0.j, -0.5 + 0.j, 0.5 + 0.j, 0.5 + 0.j],
         [-0.5 + 0.j, 0.5 + 0.j, 0.5 + 0.j, -0.5 + 0.j]]

    qc.initialize(psi, 0)
    qc.unitary(U, [0, 1])
    qc.measure([0, 1], [0, 1])
    qc.draw()

    IBMQ.load_account()
    provider = IBMQ.get_provider(hub='ibm-q', group='open', project='main')
    qcomp = provider.get_backend('ibm_nairobi')
    # running in ibm_nairobi. 4000 shots

    qc_transpiled = transpile(qc, backend=qcomp)
    job = execute(qc_transpiled, backend=qcomp, shots=4000)
    job_monitor(job)
    result = job.result()
    counts = result.get_counts(qc_transpiled)
    plot_histogram(counts)
    sum(counts.values())
    print(counts['00'] / sum(counts.values()))
    print(counts['01'] / sum(counts.values()))
    print(counts['10'] / sum(counts.values()))
    print(counts['11'] / sum(counts.values()))


def pm_kl_classical_born():
    """
    Runs PM classical protocol and plots Kullback-Leibler divergence among classical and Born probability distributions
    """

    # run experiment
    np.random.seed(0)
    shots = 10 ** 4

    qubit = qt.qubit.Qubit(np.array([(3 + 1.j * math.sqrt(3)) / 4., -0.5]))
    # P4 = {1/2|0x0|, 1/2|1x1|, 1/2|+x+|, 1/2|-x-|}
    proj = np.array([[[1, 0], [0, 0]], [[0, 0], [0, 1]], [[.5, .5], [.5, .5]], [[.5, -.5], [-.5, .5]]])
    measurement = POVM(weights=0.5 * np.array([1, 1, 1, 1]), proj=proj)

    experiment = qt.classical.prepare_and_measure_povm(shots, 4, qubit=qubit, measurement=measurement)

    # plot Kullback-Leibler divergence
    runs = experiment['probabilities']['runs']
    stats = experiment['probabilities']['stats']
    born = experiment['probabilities']['born']
    print('Stats:\np1={}, p2={}, p3={}, p4={}, pt={}'.format(stats[0], stats[1], stats[2], stats[3], np.sum(stats)))
    print('Born:\np1={}, p2={}, p3={}, p4={}, pt={}'.format(born[0], born[1], born[2], born[3], np.sum(born)))

    p1 = np.cumsum(runs[:, 0]) / (np.arange(len(runs[:, 0])) + 1)
    p2 = np.cumsum(runs[:, 1]) / (np.arange(len(runs[:, 1])) + 1)
    p3 = np.cumsum(runs[:, 2]) / (np.arange(len(runs[:, 2])) + 1)
    p4 = np.cumsum(runs[:, 3]) / (np.arange(len(runs[:, 3])) + 1)

    actual = np.vstack((p1, p2, p3, p4))
    expected = np.repeat(born.reshape(born.shape[0], 1), actual.shape[1], axis=1)

    rows, cols = actual.shape
    kl = np.zeros((cols,))
    for i in range(cols):
        kl[i] = sum(rel_entr(expected[:, i], actual[:, i]))

    plt.plot(kl, color='b')
    plt.show()
    return None


def pm_kl_classical_quantum_simulator():
    """
    Runs classical protocol and quantum simulator and plots Kullback-Leibler divergence among classical and
    quantum simulator probability distribution
    """
    shots = 10 ** 4

    qubit = qt.qubit.Qubit(np.array([(3 + 1.j * math.sqrt(3)) / 4., -0.5]))
    # P4 = {1/2|0x0|, 1/2|1x1|, 1/2|+x+|, 1/2|-x-|}
    proj = np.array([[[1, 0], [0, 0]], [[0, 0], [0, 1]], [[.5, .5], [.5, .5]], [[.5, -.5], [-.5, .5]]], dtype=complex)
    measurement = POVM(weights=0.5 * np.array([1, 1, 1, 1]), proj=proj)

    # run classical protocol
    experiment1 = qt.classical.prepare_and_measure_povm(shots, qubit=qubit, measurement=measurement)
    runs = experiment1['probabilities']['runs']
    born = experiment1['probabilities']['born']
    p1 = np.cumsum(runs[:, 0]) / (np.arange(len(runs[:, 0])) + 1)
    p2 = np.cumsum(runs[:, 1]) / (np.arange(len(runs[:, 1])) + 1)
    p3 = np.cumsum(runs[:, 2]) / (np.arange(len(runs[:, 2])) + 1)
    p4 = np.cumsum(runs[:, 3]) / (np.arange(len(runs[:, 3])) + 1)
    experimental1 = np.vstack((p1, p2, p3, p4))
    theoretical = np.repeat(born.reshape(born.shape[0], 1), experimental1.shape[1], axis=1)
    print('Classical protocol executed')

    # run quantum circuit in Qiskit simulator
    experiment2 = qt.quantum.prepare_and_measure_povm(shots, qubit, measurement)
    import collections
    memory = experiment2["memory"]
    experimental2 = np.zeros(experimental1.shape)
    for i in range(len(memory)):
        summary = collections.Counter(memory[0: i + 1])
        summary = np.array([summary[k] for k in sorted(summary.keys())])
        p = np.zeros((measurement.size(),))
        p[:summary.shape[0]] = summary / np.sum(summary)
        experimental2[:, i] = p
    print('Quantum Circuit executed')

    # plot kl divergence
    _, cols = experimental1.shape
    klte1 = np.zeros((cols,))
    klte2 = np.zeros((cols,))
    klee = np.zeros((cols,))

    for i in range(cols):
        klte1[i] = sum(rel_entr(theoretical[:, i], experimental1[:, i]))
        klte2[i] = sum(rel_entr(theoretical[:, i], experimental2[:, i]))
        klee[i] = sum(rel_entr(experimental2[:, i], experimental1[:, i]))

    plt.title('Kullback-Leibler divergence {:.0E} shots'.format(shots))
    plt.plot(klte1, color='b', label='Born vs Classical Protocol')
    plt.plot(klte2, color='r', label='Born vs Quantum Simulator')
    plt.legend()
    plt.show()
    return None


def pm_kl_classical_quantum_simulator_born(shots):
    np.random.seed(1976)
    mpl.rcParams['mathtext.fontset'] = 'stix'
    mpl.rcParams['font.family'] = 'STIXGeneral'

    fig, ax = plt.subplots(1, 1, layout='constrained')

    ax.xaxis.set_tick_params(which='major', size=5, width=1, direction='in', top='on')
    ax.xaxis.set_tick_params(which='minor', size=3, width=1, direction='in', top='on')
    ax.yaxis.set_tick_params(which='major', size=5, width=1, direction='in', right='on')
    ax.yaxis.set_tick_params(which='minor', size=3, width=1, direction='in', right='on')

    fig.suptitle(r'Cross-POVM')
    fig.supxlabel('Number of shots')
    fig.supylabel('Kullback-Leibler divergence')

    qubit = qt.qubit.Qubit(np.array([(3 + 1.j * math.sqrt(3)) / 4., -0.5]))
    proj = np.array([[[1, 0], [0, 0]], [[0, 0], [0, 1]], [[.5, .5], [.5, .5]], [[.5, -.5], [-.5, .5]]], dtype=complex)
    measurement = POVM(weights=0.5 * np.array([1, 1, 1, 1]), proj=proj)

    # run classical protocol
    experiment1 = qt.classical.prepare_and_measure_povm(shots, qubit=qubit, measurement=measurement)
    runs = experiment1['probabilities']['runs']
    stats = experiment1['probabilities']['stats']
    born = experiment1['probabilities']['born']
    p1 = np.cumsum(runs[:, 0]) / (np.arange(len(runs[:, 0])) + 1)
    p2 = np.cumsum(runs[:, 1]) / (np.arange(len(runs[:, 1])) + 1)
    p3 = np.cumsum(runs[:, 2]) / (np.arange(len(runs[:, 2])) + 1)
    p4 = np.cumsum(runs[:, 3]) / (np.arange(len(runs[:, 3])) + 1)
    experimental1 = np.vstack((p1, p2, p3, p4))
    theoretical = np.repeat(born.reshape(born.shape[0], 1), experimental1.shape[1], axis=1)
    print('Stats:p1={}, p2={}, p3={}, p4={}, pt={}'.format(stats[0], stats[1], stats[2], stats[3], np.sum(stats)))
    print('Born:p1={}, p2={}, p3={}, p4={}, pt={}'.format(born[0], born[1], born[2], born[3], np.sum(born)))
    print('Classical protocol executed')

    # run quantum circuit in Qiskit simulator
    experiment2 = qt.quantum.prepare_and_measure_povm(shots, qubit=qubit, povm=measurement)
    import collections
    memory = experiment2["memory"]
    print('Stats={}'.format(experiment2["probabilities"]))
    print('Quantum Circuit executed')

    experimental2 = np.zeros(experimental1.shape)
    for i in range(len(memory)):
        summary = collections.Counter(memory[0: i + 1])
        summary = np.array([summary[k] for k in sorted(summary.keys())])
        p = np.zeros((measurement.size(),))
        p[:summary.shape[0]] = summary / np.sum(summary)
        experimental2[:, i] = p

    # plot kl divergence
    _, cols = experimental1.shape
    klte1 = np.zeros((cols,))
    klte2 = np.zeros((cols,))
    kle1t = np.zeros((cols,))
    kle1e2 = np.zeros((cols,))

    for i in range(cols):
        klte1[i] = sum(rel_entr(theoretical[:, i], experimental1[:, i]))
        klte2[i] = sum(rel_entr(theoretical[:, i], experimental2[:, i]))
        kle1t[i] = sum(rel_entr(experimental1[:, i], theoretical[:, i]))
        kle1e2[i] = sum(rel_entr(experimental1[:, i], experimental2[:, i]))

    ax.plot(klte1, color='b', label='Born vs. Classical Protocol', linewidth=2)
    ax.plot(klte2, color='r', label='Born vs. Quantum Simulator', linewidth=2, linestyle='-', alpha=0.7)
    ax.plot(kle1e2, color='g', label='Classical Protocol vs. Quantum Simulator', linewidth=2, linestyle='-', alpha=0.7)
    ax.legend()

    plt.show()
    return None


def pm_kl_multiplot(shots):
    np.random.seed(0)
    mpl.rcParams['mathtext.fontset'] = 'stix'
    mpl.rcParams['font.family'] = 'STIXGeneral'

    cases = ['Cross-POVM', 'Trine-POVM', 'SIC-POVM', r"Random-PVM", r"Random-POVM", r"Random-POVM"]

    fig, axs = plt.subplots(3, 2, figsize=(8, 10), layout='constrained')

    for ax, title in zip(axs.flat, cases):
        ax.xaxis.set_tick_params(which='major', size=5, width=1, direction='in', top='on')
        ax.xaxis.set_tick_params(which='minor', size=3, width=1, direction='in', top='on')
        ax.yaxis.set_tick_params(which='major', size=5, width=1, direction='in', right='on')
        ax.yaxis.set_tick_params(which='minor', size=3, width=1, direction='in', right='on')
        # ax.set_ylabel(r'$D_{KL}$', labelpad=2)
        ax.set_title(title)

    fig.supxlabel('Number of shots')
    fig.supylabel('Kullback-Leibler divergence')

    qubit = qt.qubit.Qubit(np.array([(3 + 1.j * math.sqrt(3)) / 4., -0.5]))

    # EXPERIMENT 1: RANDOM-PVM
    experiment1 = qt.classical.prepare_and_measure_pvm(shots)

    qubit1 = experiment1['qubit']
    runs1 = experiment1['probabilities']['runs']
    stats1 = experiment1['probabilities']['stats']
    born1 = experiment1['probabilities']['born']
    print('EXPERIMENT 1:\nState:{}\nPVM:{}'.format(qubit1, 'RANDOM-PVM'))
    print('Stats:p1={}, p2={}, pt={}'.format(stats1[0], stats1[1], np.sum(stats1)))
    print('Born:p1={}, p2={}, pt={}'.format(born1[0], born1[1], np.sum(born1)))

    p11 = np.cumsum(runs1[:, 0]) / (np.arange(len(runs1[:, 0])) + 1)
    p12 = np.cumsum(runs1[:, 1]) / (np.arange(len(runs1[:, 1])) + 1)

    actual1 = np.vstack((p11, p12))
    expected1 = np.repeat(born1.reshape(born1.shape[0], 1), actual1.shape[1], axis=1)
    plot_kl(axs[1][1], actual1, expected1)

    # EXPERIMENT 2: CROSS-POVM
    # P4 = {1/2|0x0|, 1/2|1x1|, 1/2|+x+|, 1/2|-x-|}
    proj2 = np.array([[[1, 0], [0, 0]], [[0, 0], [0, 1]], [[.5, .5], [.5, .5]], [[.5, -.5], [-.5, .5]]])
    measurement2 = POVM(weights=0.5 * np.array([1, 1, 1, 1]), proj=proj2)
    experiment2 = qt.classical.prepare_and_measure_povm(shots, 4, qubit=qubit, measurement=measurement2)

    runs2 = experiment2['probabilities']['runs']
    stats2 = experiment2['probabilities']['stats']
    born2 = experiment2['probabilities']['born']
    print('EXPERIMENT 2:\nState:{}\nPOVM:{}'.format(qubit, 'CROSS-POVM'))
    print('Stats:p1={}, p2={}, p3={}, p4={}, pt={}'.format(stats2[0], stats2[1], stats2[2], stats2[3], np.sum(stats2)))
    print('Born:p1={}, p2={}, p3={}, p4={}, pt={}'.format(born2[0], born2[1], born2[2], born2[3], np.sum(born2)))

    p21 = np.cumsum(runs2[:, 0]) / (np.arange(len(runs2[:, 0])) + 1)
    p22 = np.cumsum(runs2[:, 1]) / (np.arange(len(runs2[:, 1])) + 1)
    p23 = np.cumsum(runs2[:, 2]) / (np.arange(len(runs2[:, 2])) + 1)
    p24 = np.cumsum(runs2[:, 3]) / (np.arange(len(runs2[:, 3])) + 1)

    actual2 = np.vstack((p21, p22, p23, p24))
    expected2 = np.repeat(born2.reshape(born2.shape[0], 1), actual2.shape[1], axis=1)
    plot_kl(axs[0][0], actual2, expected2)

    # EXPERIMENT 3: TRINE-POVM
    one = Qubit(np.array([1, 0])).rho()
    two = Qubit(0.5 * np.array([1, math.sqrt(3)])).rho()
    three = Qubit(0.5 * np.array([1, -math.sqrt(3)])).rho()
    measurement3 = POVM(weights=2. / 3 * np.array([1, 1, 1]), proj=np.array([one, two, three], dtype=complex))
    experiment3 = qt.classical.prepare_and_measure_povm(shots, qubit=qubit, measurement=measurement3)

    runs3 = experiment3['probabilities']['runs']
    stats3 = experiment3['probabilities']['stats']
    born3 = experiment3['probabilities']['born']
    print('EXPERIMENT 3:\nState:{}\nPOVM:{}'.format(qubit, 'TRINE-POVM'))
    print('Stats:p1={}, p2={}, p3={}, pt={}'.format(stats3[0], stats3[1], stats3[2], np.sum(stats3)))
    print('Born:p1={}, p2={}, p3={}, pt={}'.format(born3[0], born3[1], born3[2], np.sum(born3)))

    p31 = np.cumsum(runs3[:, 0]) / (np.arange(len(runs3[:, 0])) + 1)
    p32 = np.cumsum(runs3[:, 1]) / (np.arange(len(runs3[:, 1])) + 1)
    p33 = np.cumsum(runs3[:, 2]) / (np.arange(len(runs3[:, 2])) + 1)

    actual3 = np.vstack((p31, p32, p33))
    expected3 = np.repeat(born3.reshape(born3.shape[0], 1), actual3.shape[1], axis=1)
    plot_kl(axs[0][1], actual3, expected3)

    # EXPERIMENT 4: SIC-POVM
    one = Qubit(np.array([1, 0])).rho()
    two = Qubit(np.array([1/math.sqrt(3), math.sqrt(2/3)])).rho()
    three = Qubit(np.array([1/math.sqrt(3),
                            math.sqrt(2/3) * (math.cos(2 * math.pi/3) + 1.j * math.sin(2 * math.pi/3))])).rho()
    four = Qubit(np.array([1/math.sqrt(3),
                           math.sqrt(2/3) * (math.cos(4 * math.pi/3) + 1.j * math.sin(4 * math.pi/3))])).rho()
    measurement4 = POVM(weights=0.5 * np.array([1, 1, 1, 1]), proj=np.array([one, two, three, four], dtype=complex))
    experiment4 = qt.classical.prepare_and_measure_povm(shots, qubit=qubit, measurement=measurement4)

    runs4 = experiment4['probabilities']['runs']
    stats4 = experiment4['probabilities']['stats']
    born4 = experiment4['probabilities']['born']
    print('EXPERIMENT 4:\nState:{}\nPOVM:{}'.format(qubit, 'SIC-POVM'))
    print('Stats:p1={}, p2={}, p3={}, p4={}, pt={}'.format(stats4[0], stats4[1], stats4[2], stats4[3], np.sum(stats4)))
    print('Born:p1={}, p2={}, p3={}, p4={}, pt={}'.format(born4[0], born4[1], born4[2], born4[2], np.sum(born3)))

    p41 = np.cumsum(runs4[:, 0]) / (np.arange(len(runs4[:, 0])) + 1)
    p42 = np.cumsum(runs4[:, 1]) / (np.arange(len(runs4[:, 1])) + 1)
    p43 = np.cumsum(runs4[:, 2]) / (np.arange(len(runs4[:, 2])) + 1)
    p44 = np.cumsum(runs4[:, 3]) / (np.arange(len(runs4[:, 3])) + 1)

    actual4 = np.vstack((p41, p42, p43, p44))
    expected4 = np.repeat(born4.reshape(born4.shape[0], 1), actual4.shape[1], axis=1)
    plot_kl(axs[1][0], actual4, expected4)

    # EXPERIMENT 5: RANDOM-POVM-1
    experiment5 = qt.classical.prepare_and_measure_povm(shots, 4)

    qubit5 = experiment5['qubit']
    runs5 = experiment5['probabilities']['runs']
    stats5 = experiment5['probabilities']['stats']
    born5 = experiment5['probabilities']['born']
    print('EXPERIMENT 5:\nState:{}\nPOVM:{}'.format(qubit5, 'RANDOM-POVM-1'))
    print('Stats:p1={}, p2={}, p3={}, p4={}, pt={}'.format(stats5[0], stats5[1], stats5[2], stats5[3], np.sum(stats5)))
    print('Born:p1={}, p2={}, p3={}, p4={}, pt={}'.format(born5[0], born5[1], born5[2], born5[2], np.sum(born5)))

    p51 = np.cumsum(runs5[:, 0]) / (np.arange(len(runs5[:, 0])) + 1)
    p52 = np.cumsum(runs5[:, 1]) / (np.arange(len(runs5[:, 1])) + 1)
    p53 = np.cumsum(runs5[:, 2]) / (np.arange(len(runs5[:, 2])) + 1)
    p54 = np.cumsum(runs5[:, 3]) / (np.arange(len(runs5[:, 3])) + 1)

    actual5 = np.vstack((p51, p52, p53, p54))
    expected5 = np.repeat(born5.reshape(born5.shape[0], 1), actual5.shape[1], axis=1)
    plot_kl(axs[2][0], actual5, expected5)

    # EXPERIMENT 6: RANDOM-POVM-2
    experiment6 = qt.classical.prepare_and_measure_povm(shots, 4)

    qubit6 = experiment6['qubit']
    runs6 = experiment6['probabilities']['runs']
    stats6 = experiment6['probabilities']['stats']
    born6 = experiment6['probabilities']['born']
    print('EXPERIMENT 6:\nState:{}\nPOVM:{}'.format(qubit6, 'RANDOM-POVM-2'))
    print('Stats:p1={}, p2={}, p3={}, p4={}, pt={}'.format(stats6[0], stats6[1], stats6[2], stats6[3], np.sum(stats6)))
    print('Born:p1={}, p2={}, p3={}, p4={}, pt={}'.format(born6[0], born6[1], born6[2], born6[2], np.sum(born6)))

    p61 = np.cumsum(runs6[:, 0]) / (np.arange(len(runs6[:, 0])) + 1)
    p62 = np.cumsum(runs6[:, 1]) / (np.arange(len(runs6[:, 1])) + 1)
    p63 = np.cumsum(runs6[:, 2]) / (np.arange(len(runs6[:, 2])) + 1)
    p64 = np.cumsum(runs6[:, 3]) / (np.arange(len(runs6[:, 3])) + 1)

    actual6 = np.vstack((p61, p62, p63, p64))
    expected6 = np.repeat(born6.reshape(born6.shape[0], 1), actual6.shape[1], axis=1)
    plot_kl(axs[2][1], actual6, expected6)

    plt.show()
    return None


def plot_kl(ax, actual, expected, label='', color='b', linewidth=1, linestyle='-'):
    rows, cols = actual.shape
    kl = np.zeros((cols,))
    for i in range(cols):
        kl[i] = sum(rel_entr(expected[:, i], actual[:, i]))
    # ax.plot(kl, color=color, label=label, linewidth=linewidth, linestyle=linestyle)
    # ax.legend()
    ax.plot(kl, color=color, linewidth=linewidth, linestyle=linestyle)


def bell_sample_probabilities(shots):
    np.random.seed(0)

    a0 = Observable(Z)
    a1 = Observable(X)
    b0 = Observable(-1 / math.sqrt(2) * (X + Z))
    b1 = Observable(1 / math.sqrt(2) * (X - Z))

    alice = Qubit(a0.eigenvector(1))
    bob = (Qubit(b0.eigenvector(1)), Qubit(b0.eigenvector(-1)))

    experiment = qt.classical.bell_singlet_full(shots, alice=(a0, a1), bob=(b0, b1))
    stats = experiment['probabilities']['stats']
    born = experiment['probabilities']['born']
    print('Stats:\n{}'.format(stats))
    print('Born:\n{}'.format(born))
    return None


def bell(shots):
    np.random.seed(0)

    mpl.rcParams['mathtext.fontset'] = 'stix'
    mpl.rcParams['font.family'] = 'STIXGeneral'

    a0 = Observable(Z)
    a1 = Observable(X)
    b0 = Observable(-1 / math.sqrt(2) * (X + Z))
    b1 = Observable(1 / math.sqrt(2) * (X - Z))

    alice = Qubit(a0.eigenvector(1))
    bob = (Qubit(b0.eigenvector(1)), Qubit(b0.eigenvector(-1)))

    experiment = qt.classical.bell_singlet_full(shots, alice=(a0, a1), bob=(b0, b1))
    actual = experiment['probabilities']['stats'].T
    expected = experiment['probabilities']['born'].T

    x = np.arange(0, 5)
    y = np.arange(0, 5)

    outcomes = ["$(+1,+1)$", "$(+1,-1)$", "$(-1,+1)$", "$(-1,-1)$"]
    measurements = ["($A_0$,$B_0$)", "$(A_0,B_1)$", "$(A_1,B_0)$", "$(A_1,B_1)$"]

    fig, ax = plt.subplots(1, 2, figsize=(14, 5))

    # actual
    im = ax[0].imshow(actual, cmap='PiYG')

    ax[0].set_xticks(np.arange(len(measurements)), labels=measurements)
    ax[0].set_yticks(np.arange(len(outcomes)), labels=outcomes)

    for i in range(len(measurements)):
        for j in range(len(outcomes)):
            text = ax[0].text(j, i, round(actual[i, j], 4),
                              ha="center", va="center", color="w")

    ax[0].set_title("Classical Protocol")

    # expected
    im = ax[1].imshow(expected, cmap='PiYG')

    ax[1].set_xticks(np.arange(len(measurements)), labels=measurements)
    ax[1].set_yticks(np.arange(len(outcomes)), labels=outcomes)

    for i in range(len(measurements)):
        for j in range(len(outcomes)):
            text = ax[1].text(j, i, round(expected[i, j], 4),
                              ha="center", va="center", color="w")

    ax[1].set_title("Born\'s rule ")

    fig.tight_layout()
    plt.show()
    return None


def bell_multiplot():
    np.random.seed(0)

    mpl.rcParams['mathtext.fontset'] = 'stix'
    mpl.rcParams['font.family'] = 'STIXGeneral'

    cases = [r'$shots=10$', r'$shots=100$', r'$shots=500$',
             r'$shots=10^3$', r'$shots=5\cdot10^3$', r'$shots=10^4$',
             r'$shots=2\cdot10^4$', r'$shots=5\cdot10^4$', r'$shots=10^5$']
    outcomes = ["$(+1,+1)$", "$(+1,-1)$", "$(-1,+1)$", "$(-1,-1)$"]
    measurements = ["($A_0$,$B_0$)", "$(A_0,B_1)$", "$(A_1,B_0)$", "$(A_1,B_1)$"]

    fig, axs = plt.subplots(3, 3, figsize=(10, 10), layout='constrained')

    for ax, title in zip(axs.flat, cases):
        ax.xaxis.set_tick_params(which='major', size=5, width=1, direction='out', top='on')
        ax.xaxis.set_tick_params(which='minor', size=3, width=1, direction='out', top='on')
        ax.yaxis.set_tick_params(which='major', size=5, width=1, direction='out', right='on')
        ax.yaxis.set_tick_params(which='minor', size=3, width=1, direction='out', right='on')
        # ax.set_ylabel(r'$p_{C}(a_x, b_y | A_x, B_y)$', labelpad=2)
        ax.set_title(title)
        ax.set_xticks(np.arange(len(measurements)), labels=measurements)
        ax.set_yticks(np.arange(len(outcomes)), labels=outcomes)

    fig.supxlabel(r'$(A_x, B_y)$')
    fig.supylabel(r'$(a_x, b_y)$')
    fig.suptitle(r'$p_{C}(a_x, b_y\:|\:A_x, B_y)$')

    a0 = Observable(Z)
    a1 = Observable(X)
    b0 = Observable(-1 / math.sqrt(2) * (X + Z))
    b1 = Observable(1 / math.sqrt(2) * (X - Z))

    alice = Qubit(a0.eigenvector(1))
    bob = (Qubit(b0.eigenvector(1)), Qubit(b0.eigenvector(-1)))

    # shots = 10
    experiment1 = qt.classical.bell_singlet_full(10, alice=(a0, a1), bob=(b0, b1))
    actual1 = experiment1['probabilities']['stats'].T
    axs[0][0].imshow(actual1, cmap='PiYG')

    # shots = 100
    experiment2 = qt.classical.bell_singlet_full(100, alice=(a0, a1), bob=(b0, b1))
    actual2 = experiment2['probabilities']['stats'].T
    axs[0][1].imshow(actual2, cmap='PiYG')

    # shots = 500
    experiment3 = qt.classical.bell_singlet_full(500, alice=(a0, a1), bob=(b0, b1))
    actual3 = experiment3['probabilities']['stats'].T
    axs[0][2].imshow(actual3, cmap='PiYG')

    # shots = 1000
    experiment4 = qt.classical.bell_singlet_full(1000, alice=(a0, a1), bob=(b0, b1))
    actual4 = experiment4['probabilities']['stats'].T
    axs[1][0].imshow(actual4, cmap='PiYG')

    # shots = 5000
    experiment5 = qt.classical.bell_singlet_full(5000, alice=(a0, a1), bob=(b0, b1))
    actual5 = experiment5['probabilities']['stats'].T
    axs[1][1].imshow(actual5, cmap='PiYG')

    # shots = 10000
    experiment6 = qt.classical.bell_singlet_full(10000, alice=(a0, a1), bob=(b0, b1))
    actual6 = experiment6['probabilities']['stats'].T
    axs[1][2].imshow(actual6, cmap='PiYG')

    # shots = 20000
    experiment7 = qt.classical.bell_singlet_full(20000, alice=(a0, a1), bob=(b0, b1))
    actual7 = experiment7['probabilities']['stats'].T
    axs[2][0].imshow(actual7, cmap='PiYG')

    # shots = 50000
    experiment8 = qt.classical.bell_singlet_full(50000, alice=(a0, a1), bob=(b0, b1))
    actual8 = experiment8['probabilities']['stats'].T
    axs[2][1].imshow(actual8, cmap='PiYG')

    # shots = 100000
    experiment9 = qt.classical.bell_singlet_full(100000, alice=(a0, a1), bob=(b0, b1))
    actual9 = experiment9['probabilities']['stats'].T
    axs[2][2].imshow(actual9, cmap='PiYG')

    fig.tight_layout()
    plt.show()
    return None
\end{minted}
\newpage
\section{Simulator and Quantum Computer benchmarking}\label{section:benchmark}
The prepare-and-measure scenario selected for the comparison is as follows
\begin{equation}
\begin{split}
\ket{\Psi}&=\frac{3 + i \sqrt{3}}{4} \ket{0} - \frac{1}{2} \ket{1}\\
\mathbb{P}_4&=\{\frac{1}{2}\ket{0}\bra{0}, \frac{1}{2}\ket{1}\bra{1}, \frac{1}{2}\ket{+}\bra{+}, \frac{1}{2}\ket{-}\bra{-} \}
\end{split}
\end{equation}
A full analysis of the achievable performances when simulating such scenario with IBM Quantum simulators and computers can be found \href{https://github.com/inaki-ortizdelandaluce/qubit-communication-simulations/blob/main/tex/xls/qc_probabilities_summary.xlsx}{here}.



\begin{table}[!ht]
    \raggedleft
    \begin{tabular}{cccccccccc}
    \hline
        \textbf{RESULTS} & \textbf{'00'} & \textbf{'01'} & \textbf{'10'} & \textbf{'11' } & \textbf{} & \textbf{} & \textbf{} & \textbf{} & \textbf{} \\ \hline
        \textbf{QS-SIMULATION} & 37 & 15 & 7 & 41  & ~ & ~ & ~ & ~ & ~ \\ 
        \textbf{QC-NAIROBI} & 40 & 14 & 3 & 43  & ~ & ~ & ~ & ~ & ~ \\ 
        \textbf{QC-PERTH} & 35 & 26 & 10 & 29  & ~ & ~ & ~ & ~ & ~ \\ 
        \textbf{QC-OSLO} & 40 & 8 & 4 & 48  & ~ & ~ & ~ & ~ & ~ \\ 
        \textbf{QC-JAKARTA} & 35 & 16 & 11 & 38  & ~ & ~ & ~ & ~ & ~ \\ 
        \textbf{QC-LAGOS} & 38 & 10 & 6 & 46 & ~ & ~ & ~ & ~ & ~ \\  \hline
    \end{tabular}
    \caption{QS-QC:100 SHOTS}
\end{table}


\begin{table}[!ht]
    \raggedleft
    \begin{tabular}{cccccccccc}
    \hline
        \textbf{RESULTS} & \textbf{'00'} & \textbf{'01'} & \textbf{'10'} & \textbf{'11' } & \textbf{} & \textbf{} & \textbf{} & \textbf{} & \textbf{} \\ \hline
        \textbf{QS-SIMULATION} & 182 & 54 & 30 & 234  & ~ & ~ & ~ & ~ & ~ \\ 
        \textbf{QC-NAIROBI} & 183 & 59 & 67 & 191  & ~ & ~ & ~ & ~ & ~ \\ 
        \textbf{QC-PERTH} & 200 & 89 & 67 & 144  & ~ & ~ & ~ & ~ & ~ \\ 
        \textbf{QC-OSLO} & 189 & 52 & 33 & 226  & ~ & ~ & ~ & ~ & ~ \\ 
        \textbf{QC-JAKARTA} & 187 & 60 & 47 & 206  & ~ & ~ & ~ & ~ & ~ \\ 
        \textbf{QC-LAGOS} & 164 & 59 & 43 & 234 & ~ & ~ & ~ & ~ & ~ \\  \hline
    \end{tabular}
    \caption{QS-QC: 500 SHOTS}
\end{table}

\begin{table}[!ht]
    \raggedleft
    \begin{tabular}{cccccccccc}
    \hline
        \textbf{RESULTS} & \textbf{'00'} & \textbf{'01'} & \textbf{'10'} & \textbf{'11' } & \textbf{} & \textbf{} & \textbf{} & \textbf{} & \textbf{} \\ \hline
        \textbf{QS-SIMULATION} & 362 & 117 & 67 & 454  & ~ & ~ & ~ & ~ & ~ \\ 
        \textbf{QC-NAIROBI} & 335 & 162 & 112 & 391  & ~ & ~ & ~ & ~ & ~ \\ 
        \textbf{QC-PERTH} & 337 & 185 & 121 & 357  & ~ & ~ & ~ & ~ & ~ \\ 
        \textbf{QC-OSLO} & 379 & 100 & 52 & 469  & ~ & ~ & ~ & ~ & ~ \\ 
        \textbf{QC-JAKARTA} & 378 & 160 & 76 & 386  & ~ & ~ & ~ & ~ & ~ \\ 
        \textbf{QC-LAGOS} & 366 & 113 & 60 & 461 & ~ & ~ & ~ & ~ & ~ \\ \hline
    \end{tabular}
    \caption{QS-QC: 1000 SHOTS}
\end{table}





