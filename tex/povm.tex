We have already discussed how every qubit POVM can be written as a coarse graining of rank-1 projectors \cite{barrett2002}, such that the protocol implementation can restrict without any loss in generality to POVMs proportional to rank-1 projectors. 

Even if the final goal is to build POVMs to test how the classical protocol converges with the quantum theory for any random state and measurement, we will also discuss some general rank-1 POVMs with interesting properties, for example, 
\begin{enumerate}
    \item The measure needed in the eavesdropping  of the BB84 protocol \cite{nielsen2000}
\begin{equation}
    \mathbb{P}_4 = \{\frac{1}{2}\ket{0}\bra{0}, \frac{1}{2}\ket{1}\bra{1}, \frac{1}{2}\ket{+}\bra{+}, \frac{1}{2}\ket{-}\bra{-} \}
\end{equation}
    \item The Trine-POVM, consisting of POVM elements uniformly distributed on an equatorial plane of the Bloch sphere
\begin{equation}
\{\ket{\Psi_1}=\ket{0},\ \ket{\Psi_2}=\frac{1}{2}\ket{0} + \frac{\sqrt{3}}{2} \ket{1},\ \ket{\Psi_3}=\frac{1}{2}\ket{0} - \frac{\sqrt{3}}{2} \ket{1}\}
\end{equation}
    \item The SIC-POVMs, a well-known family of symmetric informationally complete positive operator-valued measures, which are proven to be very relevant in quantum state tomography and quantum cryptography fields among others \cite{renes2004}. The simplest SIC-POVM is the one with states the vertices of a regular tetrahedron in the Bloch sphere, see Figure \ref{fig:sic_povm}.
\end{enumerate}

The strategies to build the rank-1 POVMs and perform the measurement are different in the classical simulation protocol and in the quantum circuit model, as we will see later, so next sections will go through the methodologies applied for each case.

\begin{figure}[!ht]
\begin{center}
\centerline{\includesvg[height=4cm]{images/sic_povm.svg}}
\caption[SIC-POVM as tetrahedron in Bloch sphere]%
{\label{fig:sic_povm}%
In the Bloch sphere representation of a qubit, the states of a SIC-POVM form a regular tetrahedron with vertices $\ket{\Psi_1}=\ket{0}$, $\ket{\Psi_2}=1/{\sqrt{3}}\ket{0} + \sqrt{2/3} \ket{1}$, $\ket{\Psi_3}=1/{\sqrt{3}}\ket{0} + \sqrt{2/3} \ e^{i\frac{2\pi}{3}} \ket{1}$ and $\ket{\Psi_4}=1/{\sqrt{3}}\ket{0} + \sqrt{2/3}\ e^{i\frac{4\pi}{3}} \ket{1}$.}
\end{center}
\end{figure}

\subsubsection{Measurement in classical simulation protocols}
As described by Sent\'is et al.\ \cite{sentis2013}, the conditions under which a set of $N$ arbitrary rank-1 operators $\{E_{k}\}$ comprises a qubit POVM such that $\sum_{k=1}^{N} a_{k} E_{k} = \mathbb{1}$, can be equivalently written in a system of four linear equations
\begin{equation}
    \sum_{k=1}^{N} a_{k} = 2
\end{equation}
\begin{equation}
    \sum_{k=1}^{N} a_{k} \vec{y}_{k} = \vec{0}
\end{equation}
where $\vec{y}_{k} \in \mathbb{R}^3$ are the Bloch vectors corresponding to the qubit pure states $\ket{v_{k}}$. The existence of the set $\{a_{k}\}$ has a direct translation into a linear programming feasibility problem we would have to solve computationally.

As an example, to build a random POVM set of $N=4$ elements, we could apply the following procedure:
\begin{enumerate}
\item Assign two rank-1 operators as projective measurement elements $E_i = \ket{v_i}\bra{v_i}$ with unknown weights $\{a_i\} \text{, where}\ i=1,2$.
\item Apply the closure relation such that the third rank-1 operator is $E_3 = \mathbb{1} - \sum_{i=1}^{2}E_i$. Note that this will not be necessarily a rank-1 operator.
\item Diagonalize $E_3$ to obtain the relevant qubit states as eigenvectors $\ket{v_3}$ and $\ket{v_4}$.
\item Convert all quibit states $\ket{v_i}$ to Bloch vectors $\vec{y}_i \text{, where } i=1,2,...4$.
\item Solve the linear programming feasibility problem
\begin{equation*}
\begin{array}{ll@{}ll}
\textbf{find}  & x = \{a_1, a_2,\dots,a_N\} &\\
\textbf{subject to}& Ax = b\ \text{where column} \ A_{*k} = (\vec{y}_k, 1),\ \text{and}\ b = (\vec{0}, 2) \\
                 & x \geq 0 
\end{array}
\end{equation*}
\end{enumerate}

Provided the optimization problem is feasible, we obtain the weights $\{a_k\}$ and compute the rank-1 operators $\{E_k\}$ conforming the POVM set elements $B_k=a_k E_k$. Then we can use Equation \ref{eq:rank1_povm} to perform the following assignment
\begin{equation}
    p_k = \frac{a_k}{2}
\end{equation}
\begin{equation}
    \ket{\vec{y}_k}\bra{\vec{y}_k} = E_k
\end{equation}
which will implement the POVMs in the form required by the classical simulation protocols, i.e. $B_{k} = 2p_{k}\ket{\vec{y}_{k}}\bra{\vec{y}_{k}}$.

\subsubsection{Measurement in quantum circuit model}
One of the objectives of this project is to compare the probability distributions obtained with the classical simulation protocols, against the distributions derived when performing generalized measurements on either quantum simulators or noisy intermediate-scale quantum computers. For the sake of such comparison, we must devise a method to encode positive operator-valued measures in a quantum circuit model.
Peres \cite{peres1995}

\begin{equation}
B_{k} = \ket{v_{k}} \bra{v_{k}}
\end{equation}

\begin{equation}
\ket{w_{k}} := \ket{v_{k}} + \sum_{s=d_{Q}+1}^{N} c_{k s} \ket{u_{k}}
\end{equation}

\begin{equation}
\bra{w_{j}} \ket{w_{k}} := \bra{v_{j}} \ket{v_{k}} + \sum_{s=d_{Q}+1}^{N} c_{j s}^{\star} c_{k s} = \delta_{j k}
\end{equation}

\begin{equation}
\sum_{i=1}^{n} v_{j i}^{\star} v_{k i} + \sum_{s=d_{Q}+1}^{N} c_{j s}^{\star} c_{k s} = \delta_{j k}
\end{equation}

\begin{equation}
\sum_{k=1}^{N}\ket{v_{k}}\bra{v_{k}} = \sum_{k=1}^{N} B_{k} = \mathbb{1}
\end{equation}


\begin{equation}
\sum_{k=1}^{N}v_{k i}^{\star} v_{k j} = \delta_{ij}
\end{equation}

\begin{equation}
M = 
\begin{pmatrix}
v_{1 1} & \dots & v_{1 d_{Q}} & c_{1,d_{Q}+1} & \dots & c_{1 N} \\
v_{2 1} & \dots & v_{2 d_{Q}} & c_{2,d_{Q}+1} & \dots & c_{2 N} \\
\vdots &  & \vdots & \vdots &  & \vdots \\
v_{N1} & \dots & v_{Nd_{Q}} & c_{N,d_Q+1} & \dots & c_{NN}
\end{pmatrix}    
\end{equation}

\[\mathcal{H}^{d_Q}\ \mathcal{K}^{N}\]