We have already discussed how every qubit POVM can be written as a coarse graining of rank-1 projectors \cite{barrett2002}, such that the protocol implementation can restrict without any loss in generality to POVMs proportional to rank-1 projectors. 

Even if the final goal is to build POVMs to test how the classical protocol converges with the quantum theory for any random state and measurement, we will also discuss some general rank-1 POVMs with interesting properties, for example, 
\begin{enumerate}
    \item The measure needed in the eavesdropping  of the BB84 protocol \cite{nielsen2000}
\begin{equation}
    \mathbb{P}_4 = \{\frac{1}{2}\ket{0}\bra{0}, \frac{1}{2}\ket{1}\bra{1}, \frac{1}{2}\ket{+}\bra{+}, \frac{1}{2}\ket{-}\bra{-} \}
\end{equation}
    \item The Trine-POVM, consisting of POVM elements uniformly distributed on an equatorial plane of the Bloch sphere
\begin{equation}
\{\ket{\Psi_1}=\ket{0},\ \ket{\Psi_2}=\frac{1}{2}\ket{0} + \frac{\sqrt{3}}{2} \ket{1},\ \ket{\Psi_3}=\frac{1}{2}\ket{0} - \frac{\sqrt{3}}{2} \ket{1}\}
\end{equation}
    \item The SIC-POVMs, a well-known family of symmetric informationally complete positive operator-valued measures, which are proven to be very relevant in quantum state tomography and quantum cryptography fields among others \cite{renes2004}. The simplest SIC-POVM is the one with states the vertices of a regular tetrahedron in the Bloch sphere, see Figure \ref{fig:sic_povm}.
\end{enumerate}

The strategies to build the rank-1 POVMs and perform the measurement are different in the classical simulation protocol and in the quantum circuit model, as we will see later, so next sections will describe the methodologies applied for each case.

\begin{figure}[!ht]
\begin{center}
\centerline{\includesvg[height=4cm]{images/sic_povm.svg}}
\caption[SIC-POVM as tetrahedron in Bloch sphere]%
{\label{fig:sic_povm}%
In the Bloch sphere representation of a qubit, the states of a SIC-POVM form a regular tetrahedron with vertices $\ket{\Psi_1}=\ket{0}$, $\ket{\Psi_2}=1/{\sqrt{3}}\ket{0} + \sqrt{2/3} \ket{1}$, $\ket{\Psi_3}=1/{\sqrt{3}}\ket{0} + \sqrt{2/3} \ e^{i\frac{2\pi}{3}} \ket{1}$ and $\ket{\Psi_4}=1/{\sqrt{3}}\ket{0} + \sqrt{2/3}\ e^{i\frac{4\pi}{3}} \ket{1}$.}
\end{center}
\end{figure}

\subsubsection{Measurement in classical simulation protocols}
As described by Sent\'is et al.\ \cite{sentis2013}, the conditions under which a set of $N$ arbitrary rank-1 operators $\{E_{k}\}$ comprises a qubit POVM such that $\sum_{k=1}^{N} a_{k} E_{k} = \mathbb{1}$, can be equivalently written in a system of four linear equations
\begin{equation}
    \sum_{k=1}^{N} a_{k} = 2
\end{equation}
\begin{equation}
    \sum_{k=1}^{N} a_{k} \vec{y}_{k} = \vec{0}
\end{equation}
where $\vec{y}_{k} \in \mathbb{R}^3$ are the Bloch vectors corresponding to the qubit pure states $\ket{v_{k}}$, such that $E_k = \ket{v_k}\bra{v_k}$. The existence of the set $\{a_{k}\}$ has a direct translation into a linear programming feasibility problem we would have to solve computationally.

As an example, to build a random POVM set of $N=4$ elements, we could apply the following procedure:
\begin{enumerate}
\item Assign two rank-1 operators as projective measurement elements $E_i = \ket{v_i}\bra{v_i}$ with unknown weights $\{a_i\} \text{, where}\ i=1,2$.
\item Apply the closure relation such that the third rank-1 operator is $E_3 = \mathbb{1} - \sum_{i=1}^{2}E_i$. Note that this will not be necessarily a rank-1 operator.
\item Diagonalize $E_3$ to obtain the relevant qubit states as eigenvectors $\ket{v_3}$ and $\ket{v_4}$.
\item Convert all qubit states $\ket{v_i}$ to Bloch vectors $\vec{y}_i \text{, where } i=1,2,...4$.
\item Solve the linear programming feasibility problem
\begin{equation*}
\begin{array}{ll@{}ll}
\text{find}  & x = \{a_1, a_2,\dots,a_N\} &\\
\text{subject to}& Ax = b\ \text{where column} \ A_{*k} = (\vec{y}_k, 1),\ \text{and}\ b = (\vec{0}, 2) \\
                 & x \geq 0 
\end{array}
\end{equation*}
\end{enumerate}

Provided the optimization problem is feasible, we obtain the weights $\{a_k\}$ and compute the rank-1 operators $E_k = \ket{v_k}\bra{v_k}$ which conform the POVM set elements $\{B_k\}$ such that $B_k=a_k E_k$. Then we can use Eq. (\ref{eq:rank1_povm}) to perform the following assignment
\begin{equation}
    p_k = \frac{a_k}{2}
\end{equation}
\begin{equation}
    \ket{\vec{y}_k}\bra{\vec{y}_k} = E_k
\end{equation}
which will implement the POVMs in the form required by the classical simulation protocols, i.e. $B_{k} = 2p_{k}\ket{\vec{y}_{k}}\bra{\vec{y}_{k}}$.

\subsubsection{Measurement in quantum circuit model}\label{section:neumark}
To compare the probability distributions obtained from classical protocols with those from quantum simulators or noisy quantum computers, we must develop a technique for encoding positive operator-valued measures in a quantum circuit model. For a POVM of $N$ elements, such technique requires to create a $N\times N$ unitary matrix $U$ representing the measurement process.

Neumark's theorem \cite{neumark1940} asserts that one can extend the Hilbert space of states $\mathcal{H}$ in which the rank-1 POVM elements 
\begin{equation}\label{eq:neumark_povm}
B_k = \ket{v_{k}} \bra{v_{k}},\ \text{where}\ \sum_{k=1}^{N} B_{k} = \mathbb{1}
\end{equation}
 are defined, in such a way that there exists in the extended space $\mathcal{K}$ a set of orthogonal projectors $\Pi_{k}$ such that $B_k$
 is the result of projecting $\Pi_{k}$ from $\mathcal{K}$ into $\mathcal{H}$. Following Peres \cite{peres1995}, we can add $N-2$ extra dimensions to $\mathcal{H}$ by introducing unit vectors $\ket{u_{k}}$ othogonal to each other and to all $\ket{v_{k}}$ in Eq. (\ref{eq:neumark_povm}). Then we can build a complete orthonormal basis $\ket{w_{k}}$ in the enlarged space $\mathcal{K}$ such that
\begin{equation}
\ket{w_{k}} := \ket{v_{k}} + \sum_{s=3}^{N} c_{k s} \ket{u_{k}}
\end{equation}
\begin{equation}\label{eq:neumark_orthonormal}
\bra{w_{j}} \ket{w_{k}} := \bra{v_{j}} \ket{v_{k}} + \sum_{s=3}^{N} c_{j s}^{\star} c_{k s} = \delta_{j k}
\end{equation}
where $c_{ks}$ are the complex coefficients to be determined. Eqs. (\ref{eq:neumark_povm}) and (\ref{eq:neumark_orthonormal}) can be rewritten in index notation as 
\begin{equation}\label{eq:neumark_closure_index}
\sum_{k=1}^{N}v_{k i}^{\star} v_{k j} = \delta_{ij}
\end{equation}
\begin{equation}\label{eq:neumark_orthonormal_index}
\sum_{i=1}^{2} v_{j i}^{\star} v_{k i} + \sum_{s=3}^{N} c_{j s}^{\star} c_{k s} = \delta_{j k}
\end{equation}
According to Eq. (\ref{eq:neumark_orthonormal_index}) the following matrix $U$ is a unitary matrix which satisfies the closure property in Eq. (\ref{eq:neumark_closure_index}) and encapsulates orthonormal states in the enlarged space $\mathcal{K}$ 
\begin{equation}
U = 
\begin{pmatrix}
v_{1 1} & v_{1 2} & c_{13} & \dots & c_{1 N} \\
v_{2 1} & v_{2 2} & c_{23} & \dots & c_{2 N} \\
\vdots & \vdots & \vdots & \vdots &  \vdots \\
v_{N1} & v_{N2} & c_{N3} & \dots & c_{NN}
\end{pmatrix}
\end{equation}
By computing the complex coefficients $c_{ks}$, we can then encode any rank-1 POVM measure into a unitary matrix $U$ which can be readily used within a quantum circuit model.