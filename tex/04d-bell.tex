Most of the methods required to implement the classical simulation protocol described in Section \ref{section:protocol_bell} have already been addressed in the previous section. Hence, the underlying methodology for the generation of random states and measurements, dual Bloch vectors and shared randomness will be used without further discussion.

\subsubsection{Bell singlet state with projective measurements}

The classical protocol is applicable to Bell single states only, Alice is restricted to projective measurements with outcomes $a=\pm1$, and Bob can perform arbitrary POVMs, but we will restrict ourselves further with Bob performing arbitrary PVMs only, as per Toner and Bacon's original protocol \cite{toner2003}.

The expected joint probabilities will be computed for every possible combination of observables $A_{x}, B_{y}$, and outcomes $a_{x}, b_{y} = \pm1$. For every Bell singlet state and set of observables, we will then sample the shared randomness $\vec{\lambda}_1$, $\vec{\lambda}_2 \in \mathbb{R}^3$ following a uniform distribution and will apply steps 1 to 4 in the protocol such that for every run we get the probability for each measurement outcome. Equation (\ref{eq:prob_classic_bell}) can then be computed by just using the probabilities outcomes as weights in a random choice whose outcome gets accumulated for each shared randomness run. The accumulated random choices will lead to the final probabilities which will be then compared against the ones obtained with Equation (\ref{eq:prob_quantum_bell}).

\subsubsection{CHSH inequality}
In addition to the computation of joint probabilities, the expectation values for every duple of observables $\mathbb{E}[A_{x}, B_{y}]$ will also be calculated according to Equation (\ref{eq:bell_expected_values}). That would allow us to use Equation (\ref{eq:bell_inequality}) and try to prove through the classical protocol the breaking of Bell's inequality under a suitable set of observables by arguing that the CHSH absolute value exceeds the classical upper bound that was deduced from the hypothesis of local hidden variable model, i.e.
\begin{equation}\label{eq:chsh_inequality}
|\mathit{CHSH}| \leq 2
\end{equation}