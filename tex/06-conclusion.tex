This project has focused on exploring the fundamental limits of quantum over classical information theories in scenarios related to prepare-and-measure and Bell experiments. Specifically, we have reproduced the quantum correlations and probability distributions produced by local measurements of a quantum system, which are the basic resource of quantum information theory. We have shown that any prediction based on projective measurements on any qubit state could be simulated classically by communicating only two classical bits, and we have extended this result to a most generalised set of measurements, the positive operator-valued measures. Furthermore, we have demonstrated through computer-based experiments that the predicted probabilities can also be reproduced, to a certain degree of precision, by existing quantum simulators and noisy intermediate-scale quantum computers. Our simulations have also proven to replicate non-locality in scenarios featuring entangled states, leading to well-known results like the CHSH inequality breaking for maximally entangled states with a well chosen set of observables. 

Overall, this project has provided practical hands-on experience on fundamental quantum information concepts and and facilitated the acquisition of a functional understanding in the formulation of quantum communication protocols, thereby offering valuable insights for any future work related to quantum technologies in our professional careers.


The results of the current project provide a solid foundation for further exploration of quantum information theory. One avenue for future work could be extending the classical simulations to higher dimensional quantum prepare-and-measure scenarios, e.g. preparing qutrit states ($d_Q = 3$), given the current project was restricted to qubits ($d_Q = 2$). Another potential direction could be adapting the Bell scenario protocol to other states beyond the singlet state, as this could provide insight into the power of different types of entanglement. We could also extend this work by running classical Bell simulations where Bob can perform arbitrary POVMs. Additionally, increasing the quantum computer resources, limited by the number of shots available, and applying quantum error correction technique to the generalized measurement experiments with noisy intermediate-scale quantum computers could improve the accuracy and reliability of the results. These avenues of research would deepen our understanding of quantum information theory and could have practical applications in the development of more advanced quantum communication protocols.
