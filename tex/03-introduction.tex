\subsection{Objective}
Quantum information theory provides a framework to quantify the power of quantum theory compared to Shannon's classical communication theory \cite{shannon}. Over the last decades, the field has flourished compelled to set realistic boundaries to the promises of quantum advantages in fields like quantum communication and computing. An important feature of quantum theory lies in the statistical correlations produced by local measurements of a quantum system. The simplest example of quantum correlations are the ones produced by projective measurements on a maximally-entangled state of two qubits, also known as a Bell pair state. Such correlations are the basic resource of bipartite quantum information theory, where various equivalences are known: one shared Bell pair plus two bits of classical information can be used to teleport one  qubit and, conversely, one shared Bell pair state plus a single qubit can be used to send two bits of classical information via superdense coding. Exploring the fundamental limits of quantum over classical advantages in these scenarios is crucial, and it will be the primary objective of this work.
\par
In the early years of the current century, Ben Toner and David Bacon developed a protocol which proves that any prediction based on projective measurements on an entangled Bell pair state could be simulated by communicating only two classical bits \cite{toner2003}. Very recently, Martin Renner, Armin Tavakoli and Marco Tulio Quintino, have extended such result to a most generalised set of measurements, the positive operator-valued measures \cite{renner2022}.
Following up such generalization, we will show via computer-based experiments that a qubit transmission can be simulated classically with a total cost of two bits for any general measurement, either in a prepare-and-measure or an entanglement scenario. This will prove experimentally that the protocols described in \cite{renner2022} can be reproduced computationally using classical computers, and that the probability distributions obtained can be compared against the ones resulting from performing generalized measurements with existing quantum simulators and noisy intermediate-scale quantum computers.
\par
Before starting to describe the different computational experiments we have carried out to achieve such goal, it is worth spending next sections to introduce some preliminary concepts and notations that have been used extensively throughout this work, specifically the definition of positive operator-valued measures, the set-up of both the prepare-and-measure and Bell scenarios and the particular details of the classical simulation protocols applied to such scenarios.
\par
\subsection{Positive operator-valued measures}\label{section:povms}
Even when most of the introductory textbooks on quantum mechanics describe the measurement postulates using projective measurements only, there exists a more general and less restrictive set of measurements called Positive Operator-Valued Measures or POVMs, see \cite{nielsen2000}\cite{peres1995}. The underlying formalism behind POVMs is uniquely well adapted for some applications where the main focus is on describing the probabilities of the different measurement outcomes rather than on the post-measurement state of the system. This is of particular interest in quantum communication and quantum information, where a more comprehensive formalism for the description of the measurements is needed, and highlights how important are the results from \cite{renner2022}, where all the results are extended to POVMs without any loss of generalisation regarding the the classical simulation cost of the protocol. For reference, we define explicitely a POVM as a set $\mathbb{P}=\{B_{k}\}$ of $k=1,...,N$ positive semidefinite operators acting on a Hilbert space $\mathcal{H}$ of dimension $d_{Q}$, which satisfies the closure property $\sum_{k=1}^{N} B_{k} = \mathbb{1}$. The operator $B_{k}$ is called a POVM element, and it is associated to the outcome $k$ of the POVM. In this work we will use extensively the property which states that every qubit POVM can be written as a coarse graining of rank-1 projectors \cite{barrett2002}, such that we will restrict our POVM calculations to the case of rank-1 projectors.

\subsection{Prepare-and-measure scenario}\label{section:pm}
As for many other communication protocols we have two well-known characters playing different roles in the quantum prepare-and-measure set-up: Alice and Bob. The prepare-and-measure scenario starts with Alice preparing a random quantum state and sending it to Bob. In a general set-up, i.e. not restricted to single qubit communication, the state prepared by Alice is a state of dimension $d_Q$ described by a positive semidefinite density matrix $\rho \in \mathcal{L}( \mathbb{C}_{d_Q}), \rho \ge 0$ with unit trace $tr(\rho)=1$. Once the state is prepared and communicated by Alice, Bob then receives it and performs a random quantum measurement on it, obtaining an outcome $k$. In the case of general POVM measurements $\mathbb P=\{B_{k}\}$ as described in Section \ref{section:povms}, the probability of outcome $k$ when performing the measurement on the state $\rho$ is given by Born's rule,

\begin{equation}\label{eq:prob_quantum}
p_Q(k|\rho,\{B_{k}\}) = tr(\rho B_{k})
\end{equation}

In the context of this work, we are interested in the classical counterpart of the quantum prepare-and-measure scenario, where the probability distributions predicted by the quantum theory (\ref{eq:prob_quantum}) are reproduced classically. All these classical counterparts of existing quantum protocols, see examples in \cite{cerf2000}, \cite{toner2003} and \cite{renner2022}, require a shared randomness $\lambda$ among Alice and Bob subject to some probability function $\pi(\lambda)$ to correlate their classical communication strategies. As it is not possible to reproduce these correlations without communication, a classical message $c$ which encodes classically the quantum state $\rho$ taking its value from a $d_C$-valued alphabet $\{1,...,d_C\}$ is also required. Alice's actions are therefore described by the conditional probability $p_A(c|\rho,\lambda)$, whereas Bob's actions are similarly described by the conditional probability $p_B(k|\{B_{k}\},c,\lambda)$. If we consider both probabilities, the correlations from the classical counterpart become

\begin{equation}\label{eq:prob_classic}
p_C(k|\rho,\{B_{k}\}) = \int_{\lambda} d\lambda\ \pi(\lambda) \sum_{c=1}^{d_C} p_A(c|\rho, \lambda)\ p_B(k|\{B_{k}\}, c, \lambda)
\end{equation}

Given Equations (\ref{eq:prob_quantum}) and (\ref{eq:prob_classic}), the classical simulation would be considered successful when, for every random state and POVM, the classical probability distribution reproduces the quantum predictions, i.e.

\begin{equation}\label{eq:prob_classic_quantum}
\forall \rho, \{B_{k}\}:\quad p_C(k|\rho,\{B_{k}\}) = p_Q(k|\rho,\{B_{k}\})
\end{equation}

\subsection{Bell scenario}
In a Bell scenario, there is a bipartite quantum system of two entangled and separated qudits, one with Alice and another one with Bob. 
Alice chooses a random local measurement $A_x$ among two possible observables $\{A_0, A_1\}$, and produces an output $a_x$ according to the distribution of her measurement elements. Following the same procedure, Bob chooses his own random local measurement $B_y$, among two possible observables $\{B_0, B_1\}$, and produces an outcome $b_y$. Even if both outcomes appear random, their joint probabilities $p_{A_x,B_y}(a_x, b_y)$ are correlated. We refer to these correlations as Bell correlations. 

Similarly to the prepare-and-measure case described in Section \ref{section:pm}, it is not possible to reproduce the correlations using a classical protocol with shared random variables without allowing classical communication among Alice and Bob once they have selected their measurements, see \cite{bell1964}. The main question here is to determine how much classical communication is needed to reproduce the probability distributions.

As \cite{renner2022} shows, it is straight-forward to adapt the prepare-and-measure classical scenario to any entangled qudit-qubit state. Here Alice chooses a random local POVM on a $d_Q$-dimensional quantum system, and produces an output according to the marginal distribution of her POVM elements. Based on her output, she computes Bob's entangled qubit post-measurement state, which is sent to Bob using the prepare-and-measure scenario. Given that Bob's post-measurement qubit state is communicated by Alice using the existing prepare-and-measure classical protocol, the classical cost of the qubit transmission will be exactly the same: two bits. 

In addition, we can restrict our quantum system to a bipartite state of two qubits and local and projection valued measures. The maximally entangled states in such scenario are the well-known Bell states $\ket{\Phi_{ij}}$, as follows
\begin{equation}\label{eq:bell_states}
\begin{split}
\ket{\Phi^{+}} \vcentcolon= \frac{1}{\sqrt{2}}(\ket{00} + \ket{11})\\
\ket{\Phi^{-}} \vcentcolon= \frac{1}{\sqrt{2}}(\ket{00} - \ket{11})\\
\ket{\Psi^{+}} \vcentcolon= \frac{1}{\sqrt{2}}(\ket{01} + \ket{10})\\
\ket{\Psi^{-}} \vcentcolon= \frac{1}{\sqrt{2}}(\ket{01} - \ket{10})
\end{split}
\end{equation}
Each local projective measurements has two eigenvalues, either $\ket{a_{x}}$ or $\ket{b_{y}}$, with outcomes $a_{x}, b_{y}=\pm1$ respectively. In this scenario, the joint probabilities can be defined as
\begin{equation}\label{eq:prob_quantum_bell}
\begin{split}
p_{A_x,B_y}(a_x, b_y) = tr[\ket{a_{x}}\bra{a_{x}} \otimes \ket{b_{y}}\bra{b_{y}} \ket{\Phi_{ij}} \bra{\Phi_{ij}}]\\
\end{split}
\end{equation}

The expected values for a given set of observables $(A_{x}, B_{y})$ can be defined from the joint probabilities as follows
\begin{equation}\label{eq:bell_expected_values}
\begin{split}
\mathbb{E}[A_{x}, B_{y}] \vcentcolon= p_{A_x,B_y}(+1, +1)\,- p_{A_x,B_y}(+1, -1)\\-p_{A_x,B_y}(-1, +1)\,+ p_{A_x,B_y}(-1, -1)
\end{split}
\end{equation}
which also leads to the famous Clauser-Horne-Shimony-Holt expression
\begin{equation}\label{eq:bell_inequality}
CHSH \vcentcolon= \mathbb{E}[A_{0},B_{0}]\,+\mathbb{E}[A_{1},B_{0}]\,+\mathbb{E}[A_{0},B_{1}]\,-\mathbb{E}[A_{1},B_{1}]
\end{equation}

The classical protocol described in [RTQ22] using a Bell singlet state $\ket{\Psi^{-}}$, proves that a classical communication of one single bit is enough to reproduce the joint probabilities when Alice performs projective measurements an Bob can either perform projective or positive operator-valued measurements. Similarly as in the prepare and measure scenario, the classical probability distribution can be defined as
\begin{equation}\label{eq:prob_classic_bell}
p_C(a_{x}, b_{y}|A_{x},B_{y}) = \int_{\lambda} d\lambda\ \pi(\lambda) p_A(c|A_{x}, \lambda)\ p_B(b_{y}|B_{y}, c, \lambda)
\end{equation}

Given Equations (\ref{eq:prob_quantum_bell}) and (\ref{eq:prob_classic_bell}), the classical simulation would be considered successful when, given a bipartite singlet state $\ket{\Psi^{-}}$, for every set of observables $\{A_{x}, B_{y}\}$, the classical probability distribution reproduces the quantum predictions, i.e.

\begin{equation}\label{eq:prob_classic_quantum_bell}
\forall A_{x}, B_{y}:\quad p_C(a_{x}, b_{y}|A_{x},B_{y}) = p_{A_x,B_y}(a_x, b_y)
\end{equation}

\subsection{Classical simulation protocols}\label{section:protocols}
\subsubsection{Prepare-and-measure with one qubit}\label{section:protocol_pm}
The classical prepare-and-measure protocol proposed by Renner, Tavakoli and Quintino \cite{renner2022} is restricted to qubits ($d_Q=2$), and makes an extensive use of the geometrical properties of a qubit state in the Bloch sphere. Since mixed qubit states are convex combinations of pure states, the protocol is also restricted to the usage of pure states only. Toner and Bacon proved that making a further restriction to projective measurements, i.e. $B_{k}^{2} = B_{k}$, the classical simulation cost was upper bounded by two classical bits ($d_C=2$) \cite{toner2003}, but \cite{renner2022} generalizes the results to positive operator-valued measures with a minimal and therefore necessary classical cost of two qubits. Finally, the protocol is also restricted without any loss of generality to POVMs proportional to rank-1 projectors, following results in \cite{barrett2002}.

In Bloch notation, qubit states $\rho$ are represented by three-dimensional real normalized vectors $\vec{x} \in \mathbb{R}^{3}$, and rank-1 POVM projectors as 
\begin{equation}\label{eq:rank1_povm}
B_{k} = 2p_{k}\ket{\vec{y}_{k}}\bra{\vec{y}_{k}}
\end{equation}
where $p_{k}\ge0,\ \sum_{k=1}^{N}p_{k}=1$ and $\ket{\vec{y}_{k}}\bra{\vec{y}_{k}} = (\mathbb{1} + \vec{y}_{k} \cdot \vec{\sigma})/2$ for some normalized vector $\vec{y}_{k} \in \mathbb{R}^{3}$, such that

\begin{equation}
tr(\rho B_{k}) = p_{k}(1 + \vec{x} \cdot \vec{y}_{k}) 
\end{equation}

Two additional functions need to be defined prior to make the classical protocol steps explicit, these are the Heaviside function
\begin{equation}
H(z) =
    \begin{cases}
      1 & \text{when $z \ge 0$}\\
      0 & \text{when $z<0$}
    \end{cases} 
\end{equation}
and $\Theta(z) := z \cdot H(z)$. Under all these considerations, the protocol, as defined in \cite{renner2022} and sketched in Figure \ref{fig:msc_pm}, is literally as follows:
\begin{enumerate}
 \item Alice and Bob share two normalized vectors $\vec{\lambda}_1, \vec{\lambda}_2 \in \mathbb{R}^{3}$, which are uniformly and independently distributed on the unit radius sphere $S_2$.
 \item Instead of sending a pure qubit $\rho = (\mathbb{1} + \vec{x} \cdot \vec{\sigma})/2$, Alice prepares two bits via the formula $c_1= H(\vec{x} \cdot \vec{\lambda}_1)$ and $c_2= H(\vec{x} \cdot \vec{\lambda}_2)$ and sends them to Bob.
 \item Bob flips each vector $\vec{\lambda}_i$ when the corresponding bit $c_i$ is zero. This is equivalent to set $\vec{\lambda}^{\prime}_{i} := (-1)^{1 + c_i} \vec{\lambda}_{i}$.
 \item Instead of performing a POVM with elements $B_{k} = 2p_{k}\ket{\vec{y}_{k}}\bra{\vec{y}_{k}}$, Bob picks one vector $\vec{y}_{k}$ from the set $\{\vec{y}_{k}\}$ according to the probabilities $\{p_{k}\}$. Then he sets $\vec{\lambda} := \vec{\lambda}^{\prime}_1$ if $\lvert \vec{\lambda}^{\prime}_1 \cdot \vec{y}_{k} \rvert \ge \lvert \vec{\lambda}^{\prime}_2 \cdot \vec{y}_{k} \rvert$ and $\vec{\lambda} := \vec{\lambda}^{\prime}_2$ otherwise. Finally, Bob outputs $k$ with probability
\end{enumerate}

\begin{equation}\label{eq:prob_classic_bob}
p_B(k|\{B_{k}\},\vec{\lambda}) = \frac{p_{k}\ \Theta(\vec{y}_{k} \cdot \vec{\lambda})}{\sum_{j}^{N}p_j\ \Theta(\vec{y}_j \cdot \vec{\lambda})}
\end{equation}

\begin{figure}[tb]
\begin{center}
\begin{msc}[msc keyword=, instance width=3.6cm]{Prepare-and-measure classical protocol}
\declinst{alice}{}{Alice}
\declinst{bob}{}{Bob}
\condition*{shared randomness $\vec{\lambda}_1, \vec{\lambda}_2$}{alice,bob}
\nextlevel[3]
\action*{$c_i = H(\vec{x} \cdot \vec{\lambda_i})$}{alice}
\nextlevel[3]
\mess{$c_1, c_2 \in \{0,1\}$}{alice}{bob}
\nextlevel[1]
\action*{$\vec{\lambda}^{\prime}_i = (-1)^{1 + c_i} \vec{\lambda_i}$}{bob}
\nextlevel[3]
\action*{$p_B(k|\{B_{k}\},\vec{\lambda})$}{bob}
\nextlevel[2]
\end{msc}
\end{center}
\caption{Classical prepare-and-measure protocol sequence: Alice sends two bits $c_1, c_2$ resulting from projecting the qubit state's Bloch vector $\vec{x}$ with respect to the shared random vectors $\vec{\lambda}_1, \vec{\lambda}_2$ through a classical channel, Bob flips the shared random vectors when necessary, and finally computes classical probability outcomes according to Equation (\ref{eq:prob_classic_bob}).}
\label{fig:msc_pm}
\end{figure}

%https://tex.stackexchange.com/questions/368604/how-to-draw-a-half-and-half-colored-circle
%\pgfmathsetmacro{\radius}{2}
%\begin{tikzpicture}
%    %\fill[gray!50,opacity=0.7] (0,0) circle (\radius); 
%    %\fill[gray!10, opacity=1] (0,0) -- (135:\radius) arc (135:315:\radius) -- cycle; 
%    \draw[] circle(\radius);
%    \draw[thick,->] (0,0,0) -- (\radius,0,0) node[anchor=north west]{$\vec{\lambda}_1$};
%    \draw[thick,->] (0,0,0) -- (-0.3473, 1.9696,0) node[anchor=south east]{$\vec{\lambda}_2$};
%    \draw[thick,->] (0,0,0) -- (1.4142, 1.4142,0) node[anchor=south]{$\vec{x}$};
%    \draw[dashed] (-1.414,1.414,0) -- (1.414,-1.414,0) node[anchor=south]{};    
%\end{tikzpicture}

%\tdplotsetmaincoords{40}{0}
%\tdplotsetrotatedcoords{150}{50}{20}
%\begin{tikzpicture}[scale=2,line join=bevel,tdplot_rotated_coords,fill %opacity=1,declare function={ colorFunc(\t)=ifthenelse(\t>0,60,300);}]
%\pgfsetlinewidth{.0pt}
%    \tdplotsphericalsurfaceplot[parametricfill]{72}{36}{2}%{transparent!0}{colorFunc(180*floor(\tdplotphi/180))
%    }{}{}{}
%\end{tikzpicture}

\subsubsection{Bell with singlet state}\label{section:protocol_bell}
Toner a Bacon showed in \cite{toner2003} that only a single bit was required to classically simulate local projective measurements on a qubit pair in a singlet state $\ket{\psi^{-}}=\frac{1}{\sqrt{2}}(\ket{01} - \ket{10})$. Renner, Tavakoli and Quintino again extended such result being Alice restricted to local projective measurements with outcomes $a=\pm 1$, and Bob allowed to perform any arbitrary POVM measure \cite{renner2022}. The steps for this protocol, sketched in Figure \ref{fig:msc_bell}, are the following:
\begin{enumerate}
 \item Alice and Bob share two normalized vectors $\vec{\lambda}_1^{\prime}, \vec{\lambda}_2 \in \mathbb{R}^{3}$, which are uniformly and independently distributed on the unit radius sphere $S_2$.
 \item Instead of performing a measurement with projectors $\ket{\pm\vec{x}}\bra{\pm\vec{x}} = (\mathbb{1} + \vec{x} \cdot \vec{\sigma})/2$, Alice outputs $a = -sgn(\vec{x} \cdot \vec{\lambda}^{\prime}_1)$ and sends the bit $c = sgn(\vec{x} \cdot \vec{\lambda}^{\prime}_1) \cdot sgn(\vec{x} \cdot \vec{\lambda}_2)$ to Bob, where 
 \begin{equation}
sgn(z) =
    \begin{cases}
      1 & \text{when $z \ge 0$}\\
      -1 & \text{when $z<0$}
    \end{cases} 
\end{equation}
 \item Bob flips the vector $\vec{\lambda}_2$ if and only if $c=-1$, i.e. he sets $\vec{\lambda}^{\prime}_{2} := c \vec{\lambda}_{2}$.
 \item Same as step 4 in the previous prepare-and-measure protocol.
\end{enumerate}

\begin{figure}[tb]
\begin{center}
\begin{msc}[msc keyword=, instance width=3.6cm]{Bell with singlet state classical protocol}
\declinst{alice}{}{Alice}
\declinst{bob}{}{Bob}
\condition*{shared randomness $\vec{\lambda}_1^{\prime}, \vec{\lambda}_2$}{alice,bob}
\nextlevel[3]
\action*{$c = sgn(\vec{x} \cdot \vec{\lambda}^{\prime}_1) \cdot sgn(\vec{x} \cdot \vec{\lambda}_2)$}{alice}
\nextlevel[3]
\mess{$c \in \{-1,1\}$}{alice}{bob}
\nextlevel[1]
\action*{$\vec{\lambda}^{\prime}_2 = c \vec{\lambda_2}$}{bob}
\nextlevel[3]
\action*{$p_B(k|\{B_{k}\},\vec{\lambda})$}{bob}
\nextlevel[2]
\end{msc}
\end{center}
\caption{Classical Bell with singlet state protocol sequence: Alice sends one bit $c$ resulting from projecting the associated projective measurement's Bloch vector $\vec{x}$ with respect to the shared random vectors $\vec{\lambda}_1, \vec{\lambda}_2$ through a classical channel, Bob flips the shared random vectors when necessary, and finally computes classical probability outcomes according to the response function from the prepare-and measure protocol, see Equation (\ref{eq:prob_classic_bob}).}
\label{fig:msc_bell}
\end{figure}

It can be proved that when Alice outputs $a=+1$, $\vec{\lambda}^{\prime}_1$ and $\vec{\lambda}^{\prime}_2$ are distributed on $S_2$ according to $\rho(\vec{\lambda}^{\prime}_i) = H(-\vec{x} \cdot \vec{\lambda}^{\prime}_i)/(2\pi)$, which corresponds to a classical description of Bob's post-measurement state $-\vec{x}$ and, conversely, when Alice outputs $a=-1$, the random vectors are distributed according to a distribution corresponding to Bob's post-measurement state $\vec{x}$, and therefore Bob can apply the same response function as in the prepare-and-measure protocol described in \ref{section:protocol_pm}. 