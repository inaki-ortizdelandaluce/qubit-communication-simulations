\renewcommand{\abstractname}{Abstract}
\begin{abstract}
In the early years of the current century, a breakthrough was made in Quantum Information Theory by David Bacon and Ben Toner, who developed a quantum communication protocol showing that any prediction based on projective measurements (PVMs) on a qubit could be simulated by communicating only two classical bits. This result has been very recently extended by Martin Renner, Armin Tavakoli and Marco Tulio Quintino to positive operator-valued measures (POVMs) without any loss of generalisation, keeping the classical cost of a qubit transmission still as two bits. In this project we have simulated such extended protocol classically and compared its outcomes against a quantum simulator and a noisy intermediate-scale quantum computer, to show how two bits of communication are enough to reproduce all quantum correlations associated to arbitrary POVMs applied to any prepare-and-measure scenario. In addition, we have also reproduced the probability distributions of a Bell experiment with an entangled two-qubit pair state using a novel classical protocol also proposed by the authors. With this investigation we give rise to explore and understand computationally some fundamental limits of quantum over classical information theories.
\end{abstract}