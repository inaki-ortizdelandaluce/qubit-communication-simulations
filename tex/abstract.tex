\renewcommand{\abstractname}{Abstract}
\begin{abstract}
In the early years of the current century, a breakthrough was made in Quantum Information Theory by David Bacon and Ben Toner, who developed a quantum communication protocol showing that any prediction based on projective measurements (PVMs) on a qubit could be simulated by communicating only two classical bits. This result has been very recently extended by Martin Renner, Armin Tavakoli and Marco Tulio Quintino to positive operator-valued measures (POVMs) without any loss of generalisation and classical cost of a qubit transmission, still two bits. In this project we have simulated such extended protocol both classically and using a quantum computing framework, to show how two bits of communication are enough to reproduce all quantum correlations associated to arbitrary POVMs applied to any prepare-and-measure scenario and also to any entangled two-qubit state. With this investigation we give rise to explore and understand computationally the fundamental limits of quantum over classical advantages.
\end{abstract}